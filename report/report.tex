\documentclass[12pt,a4paper]{article}
\usepackage[utf8]{inputenc}
\usepackage[english]{babel}
\usepackage{amsmath}
\usepackage{amsfonts}
\usepackage{amssymb}
\usepackage{graphicx}
\usepackage{enumitem}
\usepackage{geometry}
\usepackage{amsthm}
\usepackage[usenames,dvipsnames]{xcolor}
\usepackage{hyperref}
\usepackage{algorithm}
\usepackage{algpseudocode}
%https://ctan.mirror.norbert-ruehl.de/macros/latex/contrib/algorithms/algorithms.pdf
%http://tug.ctan.org/macros/latex/contrib/algorithmicx/algorithmicx.pdf
\usepackage{optidef}
\usepackage{mdframed}
%fancy todo notes see https://tex.stackexchange.com/questions/9796/how-to-add-todo-notes
\usepackage[usenames,dvipsnames]{xcolor}
\usepackage{xargs}
\usepackage{xspace}
\usepackage[colorinlistoftodos,prependcaption,textsize=tiny]{todonotes}
\newcommandx{\unsure}[2][1=]{\todo[linecolor=red,backgroundcolor=red!25,bordercolor=red,#1]{#2}}
\newcommandx{\change}[2][1=]{\todo[linecolor=blue,backgroundcolor=blue!25,bordercolor=blue,#1]{#2}}
\newcommandx{\info}[2][1=]{\todo[linecolor=OliveGreen,backgroundcolor=OliveGreen!25,bordercolor=OliveGreen,#1]{#2}}
\newcommandx{\improvement}[2][1=]{\todo[linecolor=Plum,backgroundcolor=Plum!25,bordercolor=Plum,#1]{#2}}
\newcommandx{\thiswillnotshow}[2][1=]{\todo[disable,#1]{#2}}
%increase margin width for notes
\setlength{\marginparwidth}{2cm}
%end fancy to notes
\makeatletter
\def\moverlay{\mathpalette\mov@rlay}
\def\mov@rlay#1#2{\leavevmode\vtop{%
   \baselineskip\z@skip \lineskiplimit-\maxdimen
   \ialign{\hfil$\m@th#1##$\hfil\cr#2\crcr}}}
\newcommand{\charfusion}[3][\mathord]{
    #1{\ifx#1\mathop\vphantom{#2}\fi
        \mathpalette\mov@rlay{#2\cr#3}
      }
    \ifx#1\mathop\expandafter\displaylimits\fi}
\makeatother
\newcommand{\cupdot}{\charfusion[\mathbin]{\cup}{\cdot}}

%Commands for algorithmic
\renewcommand{\algorithmicrequire}{\textbf{Input:}}
\renewcommand{\algorithmicensure}{\textbf{Output:}}

%A list of macros for symbols I am not quite sure about how to display them
\newcommandx{\BQP}{BQP\xspace}
\newcommandx{\MCP}{MCP\xspace}
\newcommandx{\SDP}{SDP\xspace}
\newcommandx{\mcp}{maximum cut problem}
\newcommandx{\ImprovedCut}{ImprovedCut}
\newcommandx{\BH}{Burer heuristic}

\geometry{right=2cm, left=2.5cm, top=2cm}

\newtheoremstyle{mythm} 
{}                    
{}                    
{}                   
{}
{\bf}               
{}                      
{.8em}                       
{}

\theoremstyle{mythm}
\newtheorem{thm}{Theorem}[section]
\newtheorem{Def}[thm]{Definition}
\newtheorem{prop}[thm]{Proposition}
\newtheorem{lem}[thm]{Lemma}
\newtheorem{cor}[thm]{Corollary}
\newtheorem*{rem}{Remark}
\newtheorem*{exa}{Example}
\newtheorem*{rec}{Recall}

\begin{document}

\tableofcontents

\section{Introduction} 
Thorughout this text we consider the \mcp, which is shown to be NP-hard by \cite{Garey1974}
To tackle this problem approximation algorithms and heurisitc algorithms have been proposed.

The text divides into conceptual parts: The first part can be considered the theoretical part, in which we try to give to develop \textcolor{red}{just enough }
We begin by introducing a semidefinite relaxation for the maximum cut problem.

In section \ref{sec:GoemansWilliamson}, we give an overview of the approximation algorithm \textcolor{red}{introduced by} Goemans and Williamson. We will explain the performance
guarantee in detail. However, the proof of polynomial runtime will not be treated here, we refer to \cite{Korte2018}.

In section \ref{rankTworelaxation}, we lay the theoretical foundation for the \BH (described in section \ref{burerHeuristic}).
This includes giving a rank-two relaxation in polar coordinates, which leads to an unconstrained nonconvex optimisation problem. This relaxation has the great feature that the
number of variables is not increased, i.e. the number of variables remains the number of vertices.
This implies good scalability to large instances.
However as the function is nonconvex, we can not expect to solve this relaxation optimally.
Therefore we face a trade-off between computational runtime and a theoretical guarantee, see \cite[p. 506]{Burer2002}.
We will conduct experiments on \textcolor{red}{Mallach instance library} to investigate this trade-off. This yields data demonstrating the effectiveness.

In section \ref{burerHeuristic} we give a detailed exposition of the \BH.
Given a set of points on the unit circle, we will provide a complete description of how to generate a best possible Goemans-Williamson-type cut in a deterministic manner. 
Furthermore we will describe an algorithm to improve a cut value by means of trial and error, where the termination criterion is given by the number of successive unsuccesful
attempts to improve the cut value.
Lastly we will end the section with algorithm of the \BH.
This is the algorithm on which the experimental data is based upon.

In section \ref{BQP2MC} we show that instances of the binary quadratic programming problem can be transformed to instances of the maximum cut problem.

In the \textcolor{red}{second part} we will go into computational results.
We implemented the heuristic \textcolor{red}{(distilation from ..)} and ran it on the instance library provided by \textcolor{red}{Mallach.}
The \textcolor{red}{results} described in \textcolor{red}{section}, gather more evidence, that the rank-two relaxation \textcolor{red}{proposed } by \textcolor{red}{Burer et
al.} is highly effective.
In the \textcolor{red}{results part} we will 
\newpage

\section{A semidefinite relaxation for the Max cut problem} 
\label{sec:GoemansWilliamson} 
Throughout this text we consider undirected graphs G=(V,E) with $ V = \left[ 1:n \right] $ and $ E \subseteq \left\{ \left( i,j \right)  \in V \times V \mid i<j \right\} $.
Note that we adopt this notation to stay coherent with \cite{Burer2002} on which this text is based.
By the condition $ i < j $ there can only be one edge between the vertices $ i $ and $ j $, thus we could have just as well considered $ E \subseteq \left\{ \left\{ i,j
\right\}  \mid i,j \in V: i \neq j  \right\} $.
Let the edge weights $ w _{ ij } = w _{ ji }  $ be given such that $ w _{ ij } = 0 $ if $ \left( i,j \right) \notin E $. We then have $ w _{ ii } = 0 $ for all $ i \in V $.
We are interested in studying the weights of cuts in the graph $ G $.
Given a bipartition of $ \left( S , \overline{ S }  \right)  $ of $ V $, a cut is the set $ \left\{ e \in E \mid \left| S \cap e \right| = 1 \text{ and } \left| \overline{ S }
\cap e \right| = 1  \right\}  $. Clearly, a cut is not uniquely defined by $ \left( S , \overline{ S }  \right)  $, as $ \left( \overline{ S } , S \right)  $ generates the
same cut. 
Furthermore a cut is uniquely defined by a set $ X $, as the bipartion is uniquely given by $ \left( X, V \setminus X \right)  $.
The corresponding weight is obtained by summing the weights of the edges in the cut.
The task of finding a cut of maximum weight is called the \mcp , abbreviated by \MCP, see \cite{Korte2018}:
\begin{mdframed}[frametitle= {Maximum Weight Cut Problem}]
\begin{tabular}{ll}
Instance: &An undirected weighted graph $ G $. \\
Task: &Find a cut in $ G $ with maximum total weight.
\end{tabular}
\end{mdframed}
It is well known that the \MCP is NP-hard, see \cite{Garey1974}. Therefore there has been a lot of work on approximation algorithms and heuristics tackling the \MCP.
A groundbreaking contribution is the approximation algorithm described by Goemans and Williamson in \cite{GoemansWilliamson1995}.
In fact the \BH is based on this approximation algorithm.
Therefore we will spend the rest of this section on developing the theory to understand the algorithm by Goemans Williamson.
We will focus on the main ideas and not go into, for example proving the polynomial runtime. \\
Following the description given by \cite[p. 268 ff]{Vazirani2003}, we can motivate the formulation of the \mcp as a binary quadratic program: \\
We give a for our purposes sufficient definition of binary quadratic program.
\begin{Def}[Binary quadratic program]
Let $ B $ be either $ \left\{ 0,1 \right\}  $ or $ \left\{ -1,1 \right\}  $, and $ Q \in \mathbb{R} ^{ n \times n }  $.
A (unrestricted) binary quadratic program is of the form:
\begin{mini}
{}{x^T Q x}{}{}
\addConstraint{ x_i \in B}{}{ \quad \forall i \in  [1:n]}
\end{mini}
\end{Def} 
To every vertex $ i $ we assign a indicator variable $ x_i $ which is constrained to be in $ \left\{ -1,1 \right\}  $.
This allows us to define the cut-defining partition $ \left( S, \overline{ S }  \right)  $ by $ S := \left\{ i \mid x_i = 1 \right\}  $ and $ \overline{ S } = \left\{ i \mid
x_i = -1 \right\}  $. For any edge $ \left\{ i,j \right\}  $ in the cut we have $ i \in S $ and $ j \in \overline{ S }  $ or vice versa. Thus we have $ x_i x_j = -1 $ for
every edge $ \left\{ i,j \right\}  $ in the cut. 
On the other hand, for every edge $ \left\{ i,j \right\}  $ that is not in the cut we have $ i,j \in S $ or $ i,j \in \overline{ S }  $, implying $ x_i x_j = 1 $.
\begin{align*}
\frac{ 1 }{ 2 } \left( 1 - x_i x_j \right) = \begin{cases}
1 & \left\{ i,j \right\} \text{ in the cut defined by } $ S $ \\
0 & \text{otherwise} 
\end{cases}
\end{align*} 
This explains, why we can write the Maximum Cut Problem, abbreviated with \MCP, as the following quadratic program.
\begin{maxi}
{}{\frac{ 1 }{ 2 } \sum_{ 1 \leq i < j \leq n    } w_{ij} (1- x_ix_j) }{}{}
\label{def:mcp} 
\addConstraint{x_i}{ \in  \left\{ -1,1 \right\}  }{\quad \forall i \in [ 1:n ]}
\end{maxi}
%\todo[inline]{Formulation \ref{def:mcp} is equivalent to }
Alternatively, we can solve the following binary quadratic program as Lemma \ref{lem:01}. 
This will prove useful as we progress in the section.
\begin{mini}
{}{ \sum _{ 1 \leq i < j \leq n } w _{ ij } x_i x_j }{}{}
\label{def:bqp}
\addConstraint{\left| x_i \right| }{  = 1     }{ \quad \forall i \in  [1:n]}
\end{mini}

For the sake of convenience we will introduce some notation to denote sums, see \cite[p. 41]{Aigner2007}. Say we want to sum the even numbers between $ 0 $ and $ 100 $, we can
write $ \sum_{ k = 0 }^{ 100 } k \left[ k \text{ even}  \right]  $. The expression in brackets, which is to be multiplied, means that 
\begin{align*}
\left[ k \text{ has property E}  \right] = 
\begin{cases}
1 & \text{if } k \text{ satisfies property E} \\
0 & \text{otherwise} 
\end{cases}
\end{align*} 
This expresses the same sum as 
\begin{align*}
\sum_{ \overset{ k = 0 }{ k \text{ even} }   }^{ 100 } k \quad \text{or} \quad \sum_{ k = 1 }^{ 50 } 2k
\end{align*} 
\begin{lem}
\label{lem:01}
The solutions of \ref{def:mcp} and \ref{def:bqp} coincide.
\end{lem} 
\begin{proof}
We first observe that the constraints of \ref{def:mcp} and \ref{def:bqp} coincide so it suffices to show that the target function of the BQP is minimised if and only if the
target funciton of the MCP is maximised. The following computation uses that $ x_i x_j \in \left\{-1,1\right\}$ for all $ i,j \in \left[ 1 : n \right]  $ :
\begin{align*}
&\sum_{ 1 \leq i < j \leq n    }^{  } w _{ ij } x_i x_j 
= \sum_{  1 \leq i < j \leq n  }^{  } w _{ ij } x_i x_j \left[ x_i x_j = 1 \right]  + \sum_{ 1 \leq i < j \leq n    }^{  } w _{ ij } x_i x_j \left[ x_i x_j = -1\right] \\
=& \sum_{  1 \leq i < j \leq n  }^{  } w _{ ij } \left[ x_i x_j = 1 \right]  - \sum_{ 1 \leq i < j \leq n    }^{  } w _{ ij } \left[ x_i x_j = -1\right] 
= \sum_{  1 \leq i < j \leq n  }^{  } w _{ ij }   - 2\sum_{ 1 \leq i < j \leq n    }^{  } w _{ ij } \left[ x_i x_j = -1\right]
\end{align*} 
Using $ \sum_{ 1 \leq i < j \leq n    }^{  } w _{ ij }  $ is constant and $ \sum_{ 1 \leq i < j \leq n    }^{  } w _{ ij } (1 - x_i x_j) = 2 \sum_{ 1 \leq i < j \leq n    }^{  }
w _{ ij } \left[  x_i x_j = -1\right]  $. 
We can thus conclude, that maximising $ \frac{ 1 }{ 2 } \sum_{  1 \leq i < j \leq n   }^{  } w _{ ij } (1 - x_i x_j) $ (and thus $ \sum_{  1 \leq
i < j \leq n   }^{  } w _{ ij } (1 - x_i x_j) $ ) minimises $ \sum_{ 1 \leq i < j \leq n    }^{  } w _{ ij } x_i x_j $ and vice versa.
\end{proof}

Let us introduce some notation allowing for more concise formulation.
Consider $ W = (w _{ ij } )  _{ 1 \leq i,j \leq n }, X =(x _{ ij } )  _{ 1 \leq i,j \leq n } \in \mathbb{R} ^{ n \times n }$. We denote the sum of their entrywise product by
$ W \bullet X = \sum_{ i = 1 }^{ n } \sum_{ j = 1 }^{ n } w _{ ij } x _{ ij }  $ and vector of diagonal elements by $ \text{diag} \left( X \right) = \left( x _{ 11 } , \dots, x _{ nn } 
\right) ^{ T } \in \mathbb{R} ^{ n }  $. Further we denote the vector of all ones by $ e $ and denote a symmetric positive semidefinite matrix $ X $ by $ X \succeq 0 $ .

The following Lemma shows a useful reformulation of \ref{def:bqp}, which will form the basis of our construction of the Goemans Williamson algorithm.
\begin{lem}
\label{lem:bqp2matrixopt} 
The \BQP \ref{def:bqp} can be rewritten into the following matrix optimisation program
\begin{mini}
{}{ \frac{ 1 }{ 2 } W \bullet X }{}{}
\label{def:matoptprog} 
\addConstraint{\text{diag}(X)}{ =e }{}
\addConstraint{\text{rank}(X)}{ =1 }{}
\addConstraint{X }{ \succeq 0 }{}
\end{mini}
\end{lem} 
\begin{proof}
First of all, we are going to show that a matrix $ X $ satisfies the conditions of \ref{def:matoptprog} if and only if there exists a $ x \in \left\{ -1,1 \right\} ^{ n }  $
such that $ X = xx^T $: \\
As $ \text{rank} (X) = 1$ implies that the columns of the matrix are scalar multiples of each other, there exist $ x,y \in \mathbb{R} ^{ n } \setminus {0}  $ such that $ X =
xy^T $. \\
\textbf{Claim}: The vectors $x $ and $ y $ are necessarily linearly dependent. \\
Assume otherwise. Then the Cauchy-Schwarz inequality (CS) is strict, i.e. $ \left| \left\langle x , y \right\rangle \right| < \left\Vert x \right\Vert \left\Vert y
\right\Vert   $. Furthermore the point $ \frac{ x+y }{ 2 }  $ is linearly independent from $ x $ and $ y $. By symmetry, we only show this for $ y $: 
Given $ x,y $ linearly independent. Assume there exists $ \lambda \in \mathbb{R}  $ such that $ \frac{ x+y }{ 2 } = \lambda y \Leftrightarrow x = (2 \lambda -1) y$
contradicting $ x,y $ linearly independent. Thus $ \left| \left\langle x , \frac{ x+y }{ 2 }  \right\rangle  \right| < \left\Vert x \right\Vert \left\Vert \frac{ x+y }{ 2 }
\right\Vert  $   and
$ \left| \left\langle y ,  \frac{ x+y }{ 2 } \right\rangle  \right| < \left\Vert y \right\Vert \left\Vert \frac{ x+y }{ 2 } \right\Vert  $. 
Now we can show the claim using the following preparatory calculations.
We consider the vector $ v := y - \frac{ \left\langle y , \frac{ x+y }{ 2 } \right\rangle  }{ \left\Vert \frac{ x+y }{ 2 } \right\Vert ^{ 2 }  }  \frac{ x+y }{ 2 } $ and
calculate:
\begin{align*}
\left\langle y , v \right\rangle = \left\Vert y \right\Vert ^{ 2 } - \frac{ \left\langle y , \frac{ x+y }{ 2 } \right\rangle  }{ \left\Vert \frac{ x+y }{ 2 } \right\Vert ^{ 2
}  } \left\langle y , \frac{ x+y }{ 2 } \right\rangle \overset{ \text{CS}  }{ >} \left\Vert y \right\Vert ^{ 2 } - \frac{ 1 }{ \left\Vert \frac{ x+y }{ 2 } \right\Vert ^{ 2 }
} \left\Vert y \right\Vert ^{ 2 } \left\Vert \frac{ x+y }{ 2 } \right\Vert ^{ 2 } =0 
\end{align*} 
Furthermore we have:
\begin{align*}
\frac{ 1 }{ 2 } \left(  \left\langle x , v \right\rangle + \left\langle y , v  \right\rangle \right) &= 
\left\langle v  , \frac{ x+y }{ 2 }  \right\rangle = \left\langle y , \frac{ x+y }{ 2 }
\right\rangle - \frac{ \left\langle y , \frac{ x+y }{ 2 }  \right\rangle  }{ \left\Vert \frac{ x+y }{ 2 }  \right\Vert ^{ 2 }  } \left\langle \frac{ x+y }{ 2 } , \frac{ x+y }{
2 } \right\rangle \\
&= \left\langle y , \frac{ x+y }{ 2 } \right\rangle - \left\langle y , \frac{ x+y }{ 2 }  \right\rangle =0
\end{align*} 
Thus we have $ \left\langle x , v \right\rangle + \left\langle y , v  \right\rangle =0 $ and using the first calculation we get $ \left\langle x , v \right\rangle <0 $ and $
\left\langle y , v  \right\rangle   > 0$. 
The claim follows by observing that $ v ^T X v = v ^T x y ^T v < 0  $ contradicting $ X $ postive semidefinite.

As $ x $ and $ y $ are linearly dependent, there exist an $ \lambda \in \mathbb{R}  $ such that $ X = \lambda x x ^T $ and $ X _{ ii } = \lambda x_i^2 $ for all $ i \in \left[
1 : n \right] $. 
Using the assumption $ \text{diag} (X) =e$ yields $ 1= \lambda x _{ i } ^{ 2 }  $ for all $ i \in \left[ 1:n \right]  $. 
This implies $ \lambda \neq 0  $ and we can rewrite $ x _{ i} ^{ 2 } = \frac{ 1 }{ \lambda  }  $ for all $ i \in \left[ 1:n \right]  $.
In other words we have $ x_i \in \left\{ - \frac{ 1 }{ \sqrt{ \lambda } } , \frac{ 1 }{ \sqrt{ \lambda } }   \right\}  $ for all $ i \in \left[ 1 : n \right] $ and setting 
$ \widetilde{ x } :=  \sqrt{ \lambda   } x $ yields $ \left| \widetilde{ x_i } \right| = 1 $ for all $ i \in \left[ 1 : n \right] $ and $ X = \widetilde{ x } \widetilde{ x } ^T  $.

For the other direction: 
Let $  x \in \left\{ -1,1 \right\} ^{ n }   $ and set $ X = x x ^T  $.
We clearly have $ \text{rank} (X) =1$, also $ x_i \in \left\{ -1,1 \right\}  $ for all $ i \in \left[ 1:n \right]  $ implies $ x_i x_i = 1 $ for all $ i \in \left[ 1:n
\right]  $, which shows $ \text{diag} (X) = e $. That $ X $ is positive semidefinite follows from:
\begin{align*}
y^T X y = y^T x x^T y = \left| (x,y) \right| ^{ 2 } \geq 0 \quad \forall y \in \mathbb{R} ^{ n } 
\end{align*} 

Finally, we conclude the proof by showing that the objective functions coincide.
Let $ X = xx^T $, with $ x \in \left\{ -1,1 \right\}  $:
\begin{align*}
\frac{ 1 }{ 2 } W \bullet X = \frac{ 1 }{ 2 } \sum_{ i,j = 1 }^{ n } w_{ij} x_i x_j \overset{ \overset{ w _{ ii } = 0}{ w _{ ij } = w _{ ji } }   }{ =} \frac{ 1 }{ 2 } \sum_{ 1
\leq i <  j \leq n   }^{  }  w _{ ij } x_i x_j
\end{align*} 
\end{proof}

\begin{rem}
In the previous proof, we showed that for linearly independent $ x,y \in \mathbb{R}  ^{ n }  $ the matrix $ x y ^T  $ is not positive semidefinite, by an explicit construction.
We will explain the intuition behind the construction:
As $ x $ and $ y $ are different, by the geometric Hahn-Banach theorem, there exists a hyperplane separating the points.
However the points being different is not to sufficient to show that there exists a separating hyperplane containing the origin.
This is where the linear independence is necessary. As the construction shows, there exists a hyperplane, generated by v, that contains the origin. Containing the origin is
critical as this ensures that $ x^Tv , y^Tv \neq 0 $ and that $ x^Tv $ and $ y^Tv $ have different signs. Only then we can deduce $ v^T xy^T v < 0 $.
\end{rem} 
%\textcolor{red}{Then we have $ x \notin \mathbb{R} (x+y) $ and $ y \notin \mathbb{R} (x+y) $ as anything else directly implies linear dependance.}
%\begin{enumerate}
%\item \textcolor{red}{Find seperating hyperplane using Hahn Banach and use that it goes through the origin }
%\item \textcolor{red}{Show that there exists a functional l s.t. $ l(x) < 0 < l(y)$  }
%\item \textcolor{red}{using Riesz Theorem we get $ \left( n,v  \right) = l(v) $ \\
%This shows: $ 0 > \left( n,x \right) \left( n,y \right) = n^T xy^T n $ contradicting positive semidef
%}
%\end{enumerate}

%\begin{exa}
%Furthermore the rank constraint destroys convexity of the set of feasible solutions:
%Consider $ n = 2 $ and the rank $ 1 $ matrices 
%\begin{align*}
%M_1 =
%\begin{pmatrix}
%1 & 0 \\
%0 & 0
%\end{pmatrix} 
%\text{ and } 
%M_2 = 
%\begin{pmatrix}
%0 & 0 \\
%0 & 1
%\end{pmatrix} 
%\end{align*} 
%Then $ \frac{ 1 }{ 2 } ( M_1 + M+2) = \frac{ 1 }{ 2 } I$ is a rank $ 2 $ matrix. 
%\end{exa} 

In order to describe the algorithm by Goemans and Williamson \cite{GoemansWilliamson1995} we need to define what a semidefinite program is.
\begin{Def}[{Semidefinite program \cite[p.258]{Vazirani2003} }] 
Let $ C, D_1 , \dots, D_k \in \mathbb{R} ^{ n } $ symmetric and $ d_1 , \dots , d_k \in \mathbb{R}  $. The general semidefinite programming problem, abbreviated to \SDP, is given by
\begin{maxi}
{}{C \bullet X}{}{}
\addConstraint{D_i \bullet X }{= d_i }{\quad \forall i \in [ 1:k ]}
\addConstraint{X }{\succeq 0 }{}
\end{maxi}
\end{Def} 
The following Theorem guarantees that \SDP can be approximately solved in polynomial time.
Showing this is a key point in showing that the algorithm presented by Goemans and Williamson has polynomial runtime.
As we will not use \SDP for the heuristic refer to \cite[p. 258 ff]{Vazirani2003} or \cite[Theorem 16.10]{Korte2018} for a proof.
\begin{thm}[{ \cite[p. 259]{Vazirani2003} }]
For any $ \varepsilon > 0  $ semidefinite programs can be  solved up within an additive error of $ \varepsilon  $, in time polynomial in $ n $ and $ \log (1/ \varepsilon ) $
\end{thm} 
By setting $ C = -\frac{ 1 }{ 2 } W $ and $ D_i = e_i e_i ^T, d_i = 1 $ for all $ i \in \left[ 1:n \right]  $ we see that we can relax the matrix optimisation program \ref{def:matoptprog} into a \SDP by simply dropping the rank
constraint:
\begin{mini}
{}{ \frac{ 1 }{ 2 } W \bullet X }{}{}
\addConstraint{ \text{diag}(X)}{=e }{}
\addConstraint{ X }{\succeq 0 }{}
\end{mini}


The first big idea is to relax \BQP \ref{def:bqp} in this ingenious way.
We replace the unit scalars $ x_i $ by unit vectors $ v_i \in \mathbb{R} ^{ d } $ (where the dimension $ d >1	$ can be chosen freely). Furthermore the product $ x_i x_j $ can be interpreted as the one dimensional scalar product, we therefore replace it by $
v_i ^T v_j $, which is the standard scalar product in $ \mathbb{R} ^{ d }  $.
Noting $ v_i \in \mathbb{R} ^{ d }  $ the relaxation writes as follows:
\begin{mini}
{}{ \sum _{ 1 \leq i < j \leq n } w _{ ij } v_i ^{ T }  v_j }{}{}
\label{def:relaxedbqp} 
\addConstraint{ \left\Vert v_i \right\Vert _{ d }  }{ = 1     }{ \quad \forall i \in  [1:n]}
\end{mini}
\begin{rem}
\begin{enumerate}
\item 
We are going to show that \ref{def:relaxedbqp} is a relaxation of \ref{def:bqp}: \\
Let $ x \in \left\{ -1, 1 \right\} ^{ n } $ be a solution to \ref{def:bqp}. 
Let $ e_1 \in \mathbb{R} ^{ d } $ be the first standard basis vector and set $ v_i = x_i e_1 \in \mathbb{R} ^{ d }  $ for all $ i \in \left[ 1:n \right]  $.
Then the $ v_i $'s form a feasible solution as for every $ i \in \left[ 1:n \right]  $ we have 
$ \left\Vert v_i \right\Vert = \left| x_i \right| \left\Vert e_1 \right\Vert _{ d } = 1 $.
Furthermore values of the objective functions coincide by $ v_i ^T v_j = x_i x_j e_1 ^T e_1 = x_i ^T  x_j  $ for all $ i,j \in \left[ 1:n \right]  $.
\item Observe that for a feasible (and optimal) solution $ v_1 , \dots , v_n $ each point $ v_i $ is located on the unit sphere $ \mathbb{S} ^{ d-1 }  $ representing the
vertex $ i $ in $ G $.
\end{enumerate}
\end{rem} 
Aiming to apply Lemma \ref{lem:bqp2matrixopt}, we can change the variables $ x_i x_j $ to $ v_i ^T v_j $. Therefore $ X _{ ij }  $ is given by $ v_i ^T v_j $ instead of $ x_i
^T x_j $.
\begin{lem}[{\cite[p. 259f]{Vazirani2003}}]
We can rewrite \ref{def:relaxedbqp} into the following \SDP
\begin{mini}
{}{ \frac{ 1 }{ 2 } W \bullet X }{}{}
\label{def:relaxedsdp} 
\addConstraint{\text{diag}(X)}{ =e }{}
\addConstraint{ X }{ \succeq 0 }{}
\end{mini}
\end{lem} 
\begin{proof}
Throughout this proof we denote the $ i $-th column of a matrix $ L $ by $ L_i $. 

Let $ X $ be a feasible solution of \ref{def:relaxedbqp}. 
As $ X $ is positive semidefinite there exists a $ L \in \mathbb{R} ^{ n \times n }  $ such that $ X = L ^T L $. 
Thus we have $ X _{ ij } = L_i ^T L_j $ for all $ i,j \in \left[ 1:n \right]  $.
For $ i \in \left[ 1:n \right] $ we set $ v_i := L_i $.
As $ X $ is a feasible solution to \ref{def:relaxedsdp} we have $ 1 = X _{ ii } = L_i ^T L_i = \left\Vert v_i \right\Vert ^{ 2 } _{ n } $, which shows $ \left\Vert v_i
\right\Vert _{ n } = 1 $ for all $ i \in \left[ 1:n \right]  $. Thus the $ v_i $'s constitute a feasible solution to \ref{def:relaxedbqp} of the same objective function value:
\begin{align*}
\frac{ 1 }{ 2 } \sum_{ i,j = 1 }^{ n } w _{ ij } v_i ^T v_j = \frac{ 1 }{ 2 } W \bullet L^T L = \frac{ 1 }{ 2 } W \bullet X.
\end{align*} 

Given a feasible solution $ v_1 , \dots , v_n $	to \ref{def:relaxedbqp} we define the matrix $ L \in \mathbb{R} ^{ n \times n } $ by $ L_i := v_i $. 
Now set $ X = L^T L $, we then have $ X _{ ij } = v_i ^T v_j $. 
As the $ v_i $'s are a feasible solution we have $ X _{ ii } = \left\Vert v_i \right\Vert ^{ 2 } _{ n } =1 $ for $ i \in \left[ 1:n \right]  $.
Furthermore $ X $ is positive semidefinite since $ y^T X y = \left\Vert Ly \right\Vert ^{ 2 } _{ n } \geq 0  $ for all $ y \in \mathbb{R} ^{ n }  $, which shows that $ X $ is
a feasible solution of \ref{def:relaxedsdp} .
By construction we have that the two feasible solutions have the same value:
\begin{align*}
\frac{ 1 }{ 2 } W \bullet X = \frac{ 1 }{ 2 } \sum_{ i,j = 1 }^{ n } w _{ ij } X _{ ij } = \frac{ 1 }{ 2 } \sum_{ i,j = 1 }^{ n } w _{ ij } v_i ^T v_j
\end{align*} 
\end{proof}
\begin{rem}
As this is essential, we reiterate how to retrieve a solution of \ref{def:relaxedbqp} from a solution $ X $ of \ref{def:relaxedsdp}:
As $ X $ is a positive semidefinite matrix, using the Cholesky decomposition we compute a matrix $ L \in \mathbb{R} ^{ n \times n }  $ sucht that $ X = L^T L $. The columns of $
L $ are a solution to \ref{def:relaxedbqp}.
\end{rem} 

The algorithm presented by Goemans and Williamson can be stated as follows \cite[p.424]{Korte2018}:
We emphasise that nonnegative edge weights are required.
This requirement is essential in the proof of Lemma \ref{lem:GWalgo}.
\begin{algorithm}
\caption{Goemans Williamson algorithm}
\begin{algorithmic}[1]
%\State Solve \ref{def:bqp} approximately, i.e. find a feasible solution such that \textcolor{red}{the assosiacted value is} greater or equal 0.995 times the optimal value.
\Require A graph with nonnegative edge weights
\Ensure A cut given by the set $ S $
\State Find an approximately optimal solution $ X $ to \ref{def:bqp} 
\State Find vectors $ y_1, ..., y_n \in S^d $ such that  $ y_i^Ty_j= x _{ ij }  $ for all $ i,j \in \left[ 1:n \right]  $ (using Cholesky decomposition)
\State Choose a random point $ r \in \mathbb{S} ^{ d } $
\State Set $ S := \left\{ i \in \left[ 1:n \right]  \mid r ^T y_i \geq 0  \right\}  $
\end{algorithmic}
\end{algorithm}

As we will not solve semidefinite programs in the heuristic presented later in the text, we will for simplicity's sake, assume that we can find an optimal solution to
\ref{def:relaxedsdp}.
That means that we have an optimal solution to \ref{def:relaxedbqp}. 
This inaccuracy can be absorbed into the approximation factor; as stated in \cite[p. 260]{Vazirani2003}. 
A detailed account of how to handle nearly optimal solutions is given in \cite{Korte2018}.
Until the end of the section we will closely follow the description provided in \cite[p. 260 ff]{Vazirani2003}.
In order to approximate the \MCP we need to relate this optimum solution to \ref{def:relaxedbqp} to a cut. 
As the matrix whose columns are an optimum solution to
\ref{def:relaxedbqp} is in general not of rank $ 1 $ we are faced with a rounding problem. 
The question boils down to the following: 
How can we derive a good cut from an optimum solution to \ref{def:relaxedbqp}?

The second big idea by Goemans and Williamson, answers the question.
As the heuristic presented later on in the text is based on this idea we will describe it in detail.

Let $ v_1 , \dots , v_n $ be an optimal solution and denote the optimum value by OPT.
We have $ v_i \in \mathbb{S} ^{ d }  $ for all $ i \in \left[ 1:n \right]  $ and 
for all $ i,j \in \left[ 1:n \right]  $ we have $ \cos( \theta _{ ij }  ) = v_i ^T v_j $. 
Here $ \theta _{ ij }  $ denotes the angle between $ v_i $ and $ v_j $.
The contribution of $ v_i $ and $ v_j $ to the value of OPT is, as we simply read off from the definition of the objective function:
\begin{align*}
\frac{ 1 }{ 2 } w _{ ij } \left( 1 - \cos( \theta _{ ij }  )  \right) 
\end{align*} 
Therefore the contribution of $ v_i $ and $ v_j $ is the greater, the closer $ \theta _{ ij }  $ is to $ \pi  $ (, while considering the same $ w _{ ij }  $). 
Geometrically this means, that diametrially opposed points have the highest contribution.
This means that roughly speaking we want to seperate diametrially opposed points.
The cuts introduced in the next Definition achieve just that.
\begin{Def}[Goemans-Williamson-type cut] 
A Goemans-Williamson-type cut for a direction $ r \in \mathbb{S} ^{ n }  $ is the cut given by 
$ \left( S , \overline{ S }  \right)  $, where we set $ S := \left\{ i \in \left[ 1:n \right]  \mid  v_i ^T r \geq 0  \right\}  $ and 
$ \overline{ S }  := \left\{ i \in \left[ 1:n \right] \mid  v_i ^T r < 0  \right\} $.
\end{Def} 
As the algorithm is based on rounding the values according to a uniformly chosen Goemans-Williamson-type cut we are naturally interested in knowing what the odds of separating
two points are.
\begin{lem}
\label{lem:angle} 
Let $ n \geq 2  $ and $ v_i , v_j \in \mathbb{S} ^{ n } $ different. We divide the unit sphere into two hemispheres by a hyperplane through the origin.
The probability that $ v_i $ and $ v_j $ are seperated is given by $ P \left[ v_i \text{ and } v_j \text{ are separated}  \right] = \frac{ \theta _{ ij }  }{ \pi }   $
\end{lem} 
\begin{proof}
Consider the plane that contains $ 0,v_i $ and $ v_j $, given by $ \text{span} \left( v_i, v_j \right) $.
A hyperplane (through the origin) seperates the two points if and only if its projection onto the plane is on intersect the arc connecting $ v_i, v_j $ on the sphere.
%\todo[inline]{Insert Picture}
%\begin{tikzpicture}
%\filldraw[black] (10:2cm) circle(1pt);
%\filldraw[black] (60:2cm) circle(1pt);
%\draw (190:2cm) -- (10:2cm);
%\draw (240:2cm) -- (60:2cm);
%\draw[thick] (0cm,0cm) circle(2cm);
%\draw[green] (0,0) arc (10:60:2cm);
%\end{tikzpicture} 
As the hyperplane is chosen uniformly and the angle between $ v_i $ and $ v_j $ is given by $ \theta _{ ij }  $ the probability, that the hyperplane separates the two points is $ \frac{ \theta _{ ij }  }{ \pi }  $.
\end{proof}

\begin{lem}
\label{lem:alpha} 
The expression $ \alpha := \min_{0 < \theta \leq \pi   } \frac{ 2 }{ \pi } \frac{ \theta }{ 1- \cos( \theta )  }  $ is well defined and $ \alpha > 0.87856 $ holds.
\end{lem} 
\begin{proof}
The enumerator has a simple zero and the denominator a zero of order two. As the terms are nonnegative we have $ \lim\limits_{\theta  \to 0 } = \frac{ 2 }{ \pi  } \frac{ \theta }{ 1 - \cos( \theta )  }
= \infty $. Thus we have a contnuous function on an interval and the
As the denominator and the numerator have simple zeroes in zero, the function can be continously extended in 0. Thus we have a contnuous function on an interval and the
minimum will be attained.
We take the derivative and get
\begin{align*}
\frac{ \partial  }{ \partial \theta } \frac{ 2 }{ \pi  } \frac{ \theta }{ 1 - \cos( \theta )  } = \frac{ 2 }{ \pi } \frac{ 1 - \cos( \theta ) - \theta \sin( \theta )  }{ \left( 1 - \cos( \theta )  \right) ^{ 2 }   } 
\end{align*} 
Thus we have a critical point if and only if $ 1 = \cos( \theta ) + \theta \sin( \theta ) $
In $ \left( 0, \pi  \right]  $ there is only one solutions, namely $ 2.33112237041442 $.
By the first part this has to be a local minimum with value $ 0.878567205784851604 $.
By $ \frac{ 2 }{ \pi  } \frac{ \pi  }{ 1 - \cos( \pi )  } =2 $ the local minimum is a global minimum and $ \alpha > 0.87856 $
\end{proof}

\begin{lem}
\label{lem:GWalgo} 
The expected value of the weight of a Goemans-Williamson-type cut is greater or equal to $ \alpha \cdot \text{OPT}  $, where OPT denotes the value of an optimal solution to
\ref{def:relaxedbqp}.
\end{lem} 
\begin{proof}
Using Lemma \ref{lem:alpha} we have for all $ \theta \in \left( 0, \pi  \right]   $:
\begin{align*}
\frac{ \theta }{ \pi  } \geq \alpha \frac{ 1 - \cos( \theta )  }{ 2 }  
\end{align*} 
We denote the weight corresponding to the edge $ \left( i,j \right)  $ by $ w _{ ij }  $ and set $ w _{ ij } = 0 $ otherwise.
The expected value of a Goemans-Williamson-type cut is given by \\ $ \sum_{ 1 \leq i < j \leq n    }^{  } w _{ ij }  P \left[ v_i \text{ and } v_j \text{ are separated}  \right]
$, together with Lemma \ref{lem:angle} we can compute:
\begin{align*}
 \sum_{ 1 \leq i < j \leq n    }^{  } w _{ ij }  P \left[ v_i \text{ and } v_j \text{ are separated}  \right]
& \geq \sum_{ 1 \leq i < j \leq n    }^{  } w _{ ij }  \frac{ \theta _{ ij }  }{ \pi } \\
&\geq \frac{ \alpha }{ 2 }  \sum_{ 1 \leq i < j \leq n    }^{  } w _{ ij }  \left( 1 - \cos( \theta _{ ij }  )  \right)  \\
&= \alpha \cdot \text{OPT} 
\end{align*} 
\end{proof}
\begin{rem}
In particular theorem \ref{lem:GWalgo} implies that there exists a Goemans-Williamson-type cut with value $ 0.8785 \cdot \text{OPT} $.
\end{rem} 

Therefore we have discussed all the, for our purposes, important aspects and can state, as in \cite[Theorem 16.12]{Korte2018}:
\begin{thm}
The Goemans-Williamson  Max-Cut-Algorithm returns a set $ S $ for which the expected value of 
the associated cut is at least 0.878 times the maximum possible value.
\end{thm} 

\section{A rank-two relaxation} 
\label{rankTworelaxation} 
\todo[inline]{Write outline of first paragraph}
\todo[inline]{Type of motivation from Goemans Williamson}
Using polar coordinates we can represent any point on the unit sphere by their angle.
In other words for every $ v \in S^1 $, there exists a $ \theta \in \mathbb{R}  $ such that 
\begin{align*}
v = 
\begin{pmatrix}
\cos( \theta ) \\
\sin( \theta ) 
\end{pmatrix} 
\end{align*} 
By this idea we can represent $ n $ points $ v_1, \dots, v_n $ on the unit sphere, by a vector 
$ \theta = \left( \theta _{ 1 } , \dots, \theta _{ n }  \right) \in \mathbb{R} ^{ n }  $, such that 
the $ i $-th vector corresponds to the $ i $-th coordinate:
\begin{align}
\label{eq:polar} 
v_i = 
\begin{pmatrix}
\cos( \theta  _{ i } ) \\
\sin( \theta _{ i } ) 
\end{pmatrix} 
\quad \text{ for all } i \in \left[ 1:n \right] 
\end{align} 
Polar coordinates have the additional benefit, that using the addition theorem the inner product simplifies to:
\begin{align}
\label{eq:sp2cos} 
v_i ^{ T } v_j = \cos( \theta _{ i }  ) \cos( \theta _{ j }  ) + \sin( \theta _{ i }  ) \sin( \theta _{ j }  ) = \cos( \theta _{ i } - \theta _{ j }  ) 
\end{align} 
Next we define the following map: 
\begin{align*}
T  : \mathbb{R} ^{ n }  \to \mathbb{R} ^{ n \times n }  , \ \theta  \mapsto \left(  \theta _{ i } - \theta _{ j } \right)
_{ \substack{i \in \left[  1:n\right] \\ j \in \left[  1:n\right] } } 
\end{align*} 
For every $ \theta \in \mathbb{R} ^{ n }  $ the resulting matrix $ T \left( \theta \right)  $ is skew-symmetric, since for all $ i,j \in \left[ 1:n \right]  $
\begin{align*}
T (\theta) _{ ij } = \theta _{ i } - \theta _{ j } = - \left( \theta _{ j } - \theta _{ i }  \right) = - T (\theta) _{ ji } 
\end{align*} 
Throughout the text application of a scalar function onto a matrix, is to be understood as entrywise application of the scalar function.
We define the following function, which will turn out to be of central importance:
\begin{align*}
f : \mathbb{R} ^{ n }  \to \mathbb{R}  , \ \theta \mapsto \frac{ 1 }{ 2 } W \bullet \cos( T(\theta) ) = \frac{ 1 }{ 2 } \sum_{ i = 1 }^{ n } \sum_{ j = 1 }^{ n } w _{ ij }
\cos( \theta _{ i } - \theta _{ j }  ) 
\end{align*} 
\improvement{Read if this makes sense}
\begin{lem}
\label{prop:propertiesf} 
The function $ f $ is 
\begin{enumerate}
\item 
invariant with respect to simultaneous, uniform rotation on every component. That means for every $ \theta \in \mathbb{R} ^{ n }  $ and every  $ \tau \in \mathbb{R}  $ we have $ f (\theta) = f (\theta + \tau e) $.
\item $ 2 \pi  $-periodic with respect to each variable, i.e., for all $ \theta \in \mathbb{R} ^{ n }  $ we have $ f ( \theta ) = f ( \theta + 2 \pi e_i)  $, where $ e_i $
denotes the $ i $-th standard basis vector.
\end{enumerate}
\end{lem} 
\begin{proof}
We beginn by proving the first property. Let $ \theta \in \mathbb{R} ^{ n }  $ and $ \tau \in \mathbb{R}  $, it is sufficient to show $ T ( \theta + \tau e) = T ( \theta) $,
as this is the only term in $ f $ that depends on $ \theta $.
This holds as for any $ i,j \in \left[ 1:n \right]  $ we have $T ( \theta + \tau e) _{ ij } =  \theta _{ i } + \tau - ( \theta _{ j } + \tau ) = \theta _{ i } - \theta _{ j } = T(
\theta) _{ ij }  $.
To show the second property let $ \theta \in \mathbb{R} ^{ n }  $ and $ i \in \left[ 1:n \right]  $ and compute using $ 2 \pi  $-periodicity of the cosine:
\begin{align*}
&\cos(   T ( \theta + 2 \pi e_i) )_{ ij } =  \cos( \theta_i + 2 \pi - \theta_j ) = \cos( \theta_i - \theta_j ) = \cos( T (\theta) ) _{ ij } \text{ and} \\
&\cos(   T ( \theta + 2 \pi e_i) )_{ ji } =  \cos( \theta_j - \theta_i - 2 \pi ) = \cos( \theta_j - \theta_i ) = \cos( T (\theta) ) _{ ji }
\end{align*} 
As all other entries of $ \cos( T(\theta) )   $ remain unchanged, $ f(\theta) = f(\theta+2\pi e_i) $ is established.
\end{proof}
Recalling \ref{eq:sp2cos} and \ref{eq:polar} we see that minimising $ f $ is nothing but solving \ref{def:relaxedbqp} in polar coordinates.
This observation is of central importance and we denote the relaxation of the \MCP:
\begin{align}
\label{def:minf}
\min_{\theta \in \mathbb{R} ^{ n } } f(\theta)
\end{align} 
This point of view has favorable and unfavorable features.
\improvement{This is an unconstrained optimization problem with a nonconvex objective function.}
As $ f $ is nonconvex we have no general tool to find a global minimum or even to decide if a local minimum is a global minimum.
In general, there can be multiple local but nonglobal minima.
However, formulation \ref{def:minf} is an unconstrained minimisation problem, with a fairly easy analytical description of $ f $. 
Making use of its simplicity we compute its partial derivatives by hand.
\begin{lem}
The partial derivatives of $ f $ for any $ j \in \left[ 1 : n \right]  $ is given by
\begin{align*}
\frac{ \partial f }{ \partial \theta _{ j }  } (\theta) = \sum_{ k = 1 }^{ n } w _{ kj } \sin \left( \theta _{ k } - \theta _{ j }  \right) 
\end{align*} 
and the gradient can be written as:
\begin{align}
\label{eq:gradf} 
\nabla f(\theta) = e^T\left( W \circ \sin( T(\theta )  \right) 
\end{align} 
Here the notation $ \circ $ stands for the Hadamard product, i.e. the entrywise product of $ W $ and $ \sin( T (\theta )  $ 
\end{lem} 
\begin{proof}
We fix $ j \in \left[ 1:n \right] $ as the index of the variable for which we compute the partial derivative. To avoid ambiguity we rename the indices of $ f $ yielding:
\begin{align*}
f(\theta) = \frac{ 1 }{ 2 } \sum_{ k = 1 }^{ n } \sum_{ i = 1 }^{ n } w _{ ki } \cos( \theta_k - \theta_i ) 
\end{align*} 
From this expression we see that the term $ k=i=j $ is equal to zero and we only need to distinguish between following two cases: \\
In case of $ k = j $ and $ i \neq j  $ we have:
\begin{align*}
\frac{ \partial  }{ \partial j } w _{ ji } \cos( \theta _{ j } - \theta_i)  = -w _{ ji } \sin( \theta_j - \theta_i  ) 
\overset{ w _{ ij } = w _{ ji }  }{ \overset{ \sin \text{ odd}   }{ =}  } 
w _{ij} \sin( \theta_i - \theta_j) 
\end{align*} 
In case of $ i = j $ and $ k \neq j  $ we have:
\begin{align*}
\frac{ \partial  }{ \partial j } w _{ kj } \cos( \theta _{ k } - \theta_j)  = w _{ kj } \sin( \theta_k - \theta_j  ) 
\end{align*} 
In the case of $ k \neq j \neq i $ the derivative is zero as the term is constant with respect to $ j $. 
By linearity we therefore conclude:
\begin{align*}
\frac{ \partial  }{ \partial j } f(\theta) = \frac{ 1 }{ 2 } \left( \sum_{ k = 1 }^{ n } w _{ kj } \sin( \theta_k - \theta_j ) + \sum_{ i = 1 }^{ n } w _{ ij } \sin( \theta_k
- \theta_j )   \right) = \sum_{ k = 1 }^{ n } w _{ kj } \sin( \theta_k - \theta_j )
\end{align*} 
To justify the gradient, we only need to observe that $ \frac{ \partial  }{ \partial j } f(\theta) $ is the sum of the entries of the $ j $-th column of the matrix $ W \circ
\sin( T(\theta) )  $. This coincides with the $ j $-th entry of $ \left( W \circ \sin( T(\theta) )  \right) ^T e = \left( e^T \left( W \circ \sin( T (\theta )  \right)
\right)^T$.
\end{proof}
To be able to classify extremal points we now compute the Hessian matrix, i.e. the second derivative of $ f $.
\begin{lem}
The Hessian matrix H of $ f $ is given by 
\begin{align}
\label{def:Hessian} 
H(\theta) = W \circ \cos( T(\theta) ) - \text{diag} \left(  \left( W \circ \cos( T (\theta) ) \right) e  \right) 
\end{align} 
and the partial derivatives of second order are explicitly give by:
\begin{align*}
\frac{ \partial ^{ 2 }  }{ \partial \theta_i \partial \theta_j } f(\theta) = \begin{cases}
w _{ij} \cos( \theta_i - \theta_j )  & \text{if } i \neq j \\
- \sum_{ k \neq j  }^{ }w _{ kj } \cos( \theta_k - \theta_j )   & \text{if } i =j 
\end{cases}
\end{align*} 
\end{lem} 
\begin{proof}
Case $ i \neq j$: \\
By $ i \neq j  $ we have that $ w _{ kj } \sin( \theta_k - \theta_j  )  $ is not constant with respect to $ \theta_i $ if and only if $ k = i $. Noting that $ \frac{ \partial
}{ \partial \theta_i } w _{ ij } \sin( \theta_i - \theta_j ) = w _{ ij } \cos( \theta_i - \theta_j )   $ we get
\begin{align*}
\frac{ \partial ^{ 2 }  }{ \partial \theta_i \partial \theta_j } f (\theta) = w _{ ij } \cos( \theta_i - \theta_j ) 
\end{align*} 
Case $ i = j $: \\
In this case we have $ w _{ jj } = 0 $ and $ \frac{ \partial  }{ \partial \theta_j } \sin( \theta_k - \theta_j  ) = - \cos( \theta_k - \theta_j ) $. By linearity we thus get
\begin{align*}
\frac{ \partial ^{ 2 }  }{ \partial \theta_j \partial \theta_j } f(\theta) = \sum_{ k \neq j   }^{  } -w _{ kj } \cos( \theta_k - \theta_j ) 
\end{align*} 
\end{proof}

Next we want to investigate the relationship between the function $ f $ and the cuts of a graph.
\begin{Def}[Angular representation of a cut]
  A vector $ \theta \in \mathbb{R} ^{ n }  $ is called an angular representation of a cut, or simply a cut, if there exist integers $ k _{ ij }  $ such that $ \theta_i - \theta_j = k _{ ij }
\pi$ for all $ i,j \in \left[ 1:n \right]  $ 
\end{Def} 
Let $ \overline{ \theta }  $ be an angular representation of a cut.
Using that $ \cos( \overline{ \theta }  _{ i } - \overline{ \theta} _{ j }  )  $ evaluates to $ 1 $ if $ k _{ ij }  $ is even and to $ -1 $ if $ k _{ij} $ is odd,
there exists a binary vector $ \overline{ x } \in \left\{ -1,1 \right\} ^{ n }  $  such that
\begin{align}
\label{eq:cutrel} 
\cos( \overline{ \theta } _{ i } - \overline{ \theta }  _{ j } ) = \overline{ x } _{ i } \overline{ x } _{ j } = \pm 1 \quad \text{for all } i,j \in \left[ 1:n \right].
\end{align} 
This can be seen by setting $ x_1 = 1 $ and 
\begin{align*}
\overline{ x }  _{ i } = \begin{cases}
1 & \text{if } \overline{ \theta}_i - \overline{ \theta}_1 \in 2 \pi \mathbb{Z} \\
-1 & \text{otherwise} 
\end{cases}
\quad \text{for } i \in \left[ 2:n \right] 
\end{align*} 
This construction is correct, as for any $ i,j \in \left[ 1:n \right]  $ we have  $ k _{
ij }  $ is even if and only if $ k _{ i1 }  $ and $ k _{ j1 } $ are both even or both odd if and only if $ \overline{ x } _i \overline{ x } _j = 1 $.
The equivalence holds by
\begin{align*}
 k _{ ij } \pi = \overline{ \theta} _{ i } - \overline{ \theta} _{ j } = \overline{ \theta} _{ i } - \overline{ \theta} _{ 1 } - ( \overline{ \theta} _{ j
} - \overline{ \theta} _{ 1 } ) = ( k _{ i1 } - k _{ j1 } ) \pi 
\end{align*} 
Then $ \overline{ x }  $ can be viewed as the cut corresponding to $ \overline{ \theta }  $.
Moreover, the cut value corresponding to $ \overline{ \theta }  $ is
\begin{align}
\label{eq:psiRelaxedCut} 
\psi ( \overline{ \theta } ) = \frac{ 1 }{ 2 } \sum_{ 1 \leq i < j \leq n  }^{  } w _{ ij } \left( 1 - \cos( \overline{ \theta } _{ i } - \overline{ \theta } _{ j }  )  \right) 
\end{align} 
In the next Proposition we give a detailed acount of this correspondence.
\begin{prop}
Modulo the uniform rotation and the periodicity for each variable, there is an one-to-one correspondence between binary cuts and angular representations
of a cut, given by
\begin{align}
\label{eq:ass1to1} 
\overline{ \theta } _{ i } = \begin{cases}
0 & \text{if } \overline{ x } _{ i } = 1 \\
\pi & \text{if } \overline{ x } _{ i } = -1
\end{cases}
\end{align} 
In the same manner we can set
\begin{align*}
\overline{ x } _{ i } = \begin{cases}
1 & \text{if } \overline{ \theta } _{ i } = 0 \\
-1 & \text{if } \overline{ \theta } _{ i  } = \pi
\end{cases}
\end{align*} 
\end{prop} 
\begin{proof}
It is clear that assignment \ref{eq:ass1to1} assigns exactly one angular representation to a binary cut.
On the other hand every angular representation of a cut has a representant, where every variable is either $ 0 $ or $ \pi $.
Let $ \overline{ \theta}   $ be an angular representation of a cut. 
By the second property, we can assume $ \overline{ \theta}_i  \in \left[ 0, 2 \pi \right)  $ for all $ i \in \left[ 1:n \right]  $.
We only need to show that there exists a $ \tau \in \mathbb{R} $ such that $ \overline{ \theta }_i - \tau \in \left\{ 0, \pi  \right\}	$ for all $ i \in \left[ 1:n \right] $.
If $ \overline{ \theta } _{ i } \in \left\{ 0, \pi  \right\}  $ for all $ i \in \left[ 1:n \right]  $ we simply pick $ \tau = 0 $ and are done.
Otherwise there exists an $ i \in \left[ 1:n \right]  $, such that $ \overline{ \theta  }_i \notin \left\{ 0, \pi  \right\}  $. 
If $ \overline{ \theta }_i \in \left( 0 , \pi  \right)  $ we choose $ \tau := \overline{ \theta }_i  $, else we have $ \overline{ \theta } _{ i } \in \left( \pi , 2 \pi
\right)  $ and we choose $ \tau := \overline{ \theta } _{ i } - \pi $.
Now let $ j \in \left[ 1:n \right]  $. 
With this choice we just have to show that 
\begin{align}
\label{eq:thetajtau} 
 \overline{ \theta } _{ j } = \begin{cases}
 \tau  & \text{if } \overline{ \theta } _{ j } \in \left( 0, \pi \right) \\
 \pi + \tau  & \text{if } \overline{ \theta } _{ j } \in \left( \pi, 2 \pi \right) 
 \end{cases}
\end{align} 
holds.
As $ \overline{ \theta }  $ is an angular representation of a cut there exists a $ k \in \mathbb{Z} $ such that 
$ \overline{ \theta } _{ i } - \overline{ \theta } _{ j } = k \pi   $. By rearranging the terms there exists a $ \widetilde{ k } \in \mathbb{Z} $ such that  
$ \overline{ \theta } _{ j } = \overline{ \theta } _{ i } - k \pi = \widetilde{ k } \pi + \tau $.
This shows equation \ref{eq:thetajtau}. 
This shows $ \left( \overline{ \theta } - \tau e  \right) _{ i } \in \left\{ 0, \pi  \right\}  $ for all $ i \in \left[ 1:n \right]  $.
Now let $ \overline{ \theta }  $ be any angular representaion of a cut and $ \overline{ \vartheta } \in \left\{ 0, \pi  \right\} ^{ n }   $ its representant.
In the proof of Proposition \ref{prop:propertiesf} we have seen that $ \cos( \overline{ \theta} _{ i } - \overline{ \theta }  _{ j }    )  = \cos( \overline{ \omega } _{ i } -
\overline{ \omega } _{ j }  ) $ for all 
\end{proof}
Using this correspondence we can represent a cut by both $ \overline{ \theta }  $ and $ \overline{ x }  $. Given a binary representation of a cut $ \overline{ x }  $ (or an
angular representation $ \overline{ \theta }  $), we denote the angular representation by $ \theta ( \overline{ x } ) $ 
(or the binary representation by $ x ( \overline{ \theta } ) $ of that same cut.
The next Proposition states an import property of angular representations of a cut.
\begin{prop}
\label{prop:cutsArestationaryPoints} 
Every angular representation of a cut $ \overline{ \theta } \in \mathbb{R} ^{ n }  $ is a stationary point of the function $ f $.
\end{prop} 
\begin{proof}
As $ \overline{ \theta } $ is a cut we have $ \overline{ \theta } _{ i } - \overline{ \theta } _{ j } \in \mathbb{Z} \pi  $ for all $ i,j \in \left[ 1:n \right]  $. 
This directly implies that every entry of $ \sin( T ( \overline{ \theta } ) )  $ is zero. Plugging this into \ref{eq:gradf} yields $ \nabla f ( \overline{ \theta } ) = 0$.
This shows that $ \overline{ \theta }  $ is a stationary point.
\end{proof}
\begin{Def}[Nonnegatively summable matrix]
A matrix $ M \in \mathbb{R} ^{ n \times n }  $ is called nonnegatively summable if the sum of the entries in every principal submatrix of $ M $ is nonnegative, or
equivalently, if $ u ^{ T} M u \geq 0    $ for every binary vector $ u \in \left\{ 0,1 \right\} ^{ n }  $ 
\end{Def} 
Put in words a principal submatrix is a matrix that we get by crossing out rows and corresponding columns, i.e. crossing out the $ i$-th row implies crossing out the $ i
$-th coloum as well and vice versa.
By definition, every semidefinite matrix is nonnegatively summable. However the other implication does not hold as the following example shows.
\begin{exa}
Let $ n>1 $ and denote the identity matrix as $ I \in \mathbb{R} ^{ n \times n }  $.
The matrix $ ee^T - I $ is nonnegatively summable but not positive semidefinite.
As all the entries of $ ee^T - I $ are nonnegative the matrix is clearly nonnegatively summable.
To show that $ ee^T - I $ is not positive semidefinite we consider the vector $ u := e_1 - e_n $. By $ n>1  $ we have $ 1 \neq n  $ and we get 
$ u ^T \left( ee^T - I \right) u = u^T \left( -1,0, \cdots, 0,1 \right) = -2 < 0  $.
\end{exa} 
Before we can provide a characterisation for maximum (and minimum) cuts in Lemma \ref{lem:M(x)nonnegsum} we prove a little calculation rule:
\begin{lem}
\label{lem:calcrule} 
Let $ W \in \mathbb{R} ^{ n \times n }  $ be a symmetric matrix, $ x \in \mathbb{R} ^{ n } $ and $ \delta,  \delta ' \in \left\{ 0,1 \right\} ^{ n }  $.
Then we have
\begin{align*}
\left( \delta \circ x \right) ^T W \left(  \delta '  \circ x \right) = \delta ^T \left( W \circ x x ^T  \right)  \delta ' =  \delta '^T \left( W \circ x x ^T  \right) \delta
\end{align*} 
\end{lem} 
\begin{proof}
First we observe that for any $ \delta \in \left\{ 0,1 \right\} ^{ n }  $ we have
\begin{align*}
\left( \delta \circ x \right) _{ i } = \begin{cases}
0 & \text{if } \delta_i = 0\\
x_i & \text{if } \delta_i = 1
\end{cases}
= \delta_i x_i
\end{align*} 
Using this we can compute:
\begin{align*}
\left( \delta \circ x \right) ^T W \left( \delta ' \circ x \right) 
&= \sum_{ i = 1 }^{ n } \left( \delta \circ x \right) _{ i } \sum_{ j = 1 }^{ n } w _{ ij } \left( \delta ' \circ x \right) _{ j } 
= \sum_{ i = 1 }^{ n } \delta _{ i } x_i \sum_{ j = 1 }^{ n } w _{ ij } \delta ' _{ j } x_j \\
&= \sum_{ i = 1 }^{ n } \delta_i \sum_{ j = 1 }^{ n } w _{ ij } x_i x_j \delta'_j = \delta ^T A \delta '
\end{align*} 
where $ A _{ ij } = w _{ ij } x_i x_j $ for all $ i,j \in \left[ 1:n \right]  $.
Thus we have $ A = W \circ xx ^T  $ and the first equality holds.
Having established the first equality the second follows by noting that $ W $ is symmetric
\begin{align*}
\left( \delta \circ x \right) ^T W  \left( \delta ' \circ x \right)  
= \left( \delta' \circ x \right) ^T W ^T \left( \delta \circ x \right) 
= \delta'^T \left( W \circ xx^T \right)  \delta
\end{align*} 
\end{proof}
\begin{lem}
\label{lem:M(x)nonnegsum} 
Let $ \overline{ x } \in \left\{ -1,1 \right\} ^{ n }  $ be given and consider the matrix $ M( \overline{ x } ) \in \mathbb{R} ^{ n \times n }  $ defined as 
\begin{align}
\label{def:M(x)} 
M ( \overline{ x }  ) = W \circ \left( \overline{  x} \overline{ x } ^{ T }  \right) - \text{diag} \left( \left( W \circ \left( \overline{ x } \overline{ x } ^T  \right)
\right) e  \right) 
\end{align} 
Then, $ \overline{ x }  $ is a maximum (respectively, minimum) cut if and only if $ M ( \overline{ x }  )  $ (respectively, $ -M ( \overline{ x }  )  $ )
is nonnegatively summable.
\end{lem} 
\begin{proof}
Consider the quadratic function $ q : \mathbb{R} ^{ n }  \to \mathbb{R}  , \ x \mapsto x^T W x /2 $.
By Lemma \ref{lem:01} we know that $ \overline{ x } \in \left\{ -1,1 \right\} ^{ n }  $ is a maximum cut if and only if $ \overline{ x }  $ satisfies  $ q \left( \overline{ x }  \right) \leq q \left( x \right)   $
for all $ x \in \left\{ -1,1 \right\} ^{ n }  $.
Consider a maximum cut $ \overline{ x } \in \left\{ -1,1 \right\} ^{ n }  $ and an arbitrary $ x \in \left\{ -1,1 \right\} ^{ n }  $.
As $ W $ is symmetric $ \overline{ x } ^{ T } W x = x^T W \overline{ x } $ holds.
Using this and the previous equivalence we have:
\begin{align}
\label{eq:nonnegsum1} 
\begin{split}
0 &\leq 
q \left( x   \right) - q \left( \overline{ x }  \right) = \frac{ 1 }{ 2 } \left( x^TWx - \overline{ x } ^{ T } W \overline{ x }  \right) = 
 \frac{ 1 }{ 2 } \left( x^TWx + \overline{ x } ^T Wx - x^TW \overline{ x } -  \overline{ x } ^{ T } W \overline{ x }  \right) \\
&= \frac{ 1 }{ 2 } \left( x + \overline{ x }  \right) ^T W \left( x - \overline{ x }  \right)
= \left( \overline{ x } + \frac{ 1 }{ 2 } ( x - \overline{ x }   \right) ^T W \left( x - \overline{ x }  \right) \\
&= \overline{ x } ^T W \left( x- \overline{ x }  \right) + \frac{ 1 }{ 2 } \left( x - \overline{ x }  \right) ^T W \left( x - \overline{ x }  \right) 
\end{split}
\end{align} 
As we have $ \overline{ x } , x \in \left\{ -1,1 \right\} ^{ n }  $ we know that either $ x_i = \overline{ x } _{ i }  $ or $ x_i = - \overline{ x } _{ i }  $ for all $ i \in
\left[ 1:n \right]  $. With this observation we see that 
\begin{align*}
\left( x - \overline{ x }  \right) _{ i } = \begin{cases}
0 & \text{ if }   x_i = \overline{ x } _{ i }   \\
- 2 \overline{ x } _i & \text{ if }   x_i \neq \overline{ x } _{ i }   
\end{cases}
\text{ for all } i \in \left[ 1:n \right] 
\end{align*} 
and by defining $ \delta \in \mathbb{R} ^{ n }  $ as
\begin{align*}
\delta_i = \begin{cases}
0 & \text{ if } x_i = \overline{ x_i } \\
1 & \text{ if } x_i \neq \overline{ x_i }  
\end{cases}
\end{align*} 
we get the identity 
\begin{align*}
x - \overline{ x } = -2 \delta \circ \overline{ x } 
\end{align*} 
With this we continue computation \ref{eq:nonnegsum1}:
\begin{align*}
0 
&\leq \overline{ x } ^T W \left( x- \overline{ x }  \right) + \frac{ 1 }{ 2 } \left( x - \overline{ x }  \right) ^T W \left( x - \overline{ x }  \right) \\
&= -2 \overline{ x } ^T W \left( \delta \circ \overline{ x }  \right) + 2 \left( \delta \circ \overline{ x }  \right) ^T W \left( \delta \circ \overline{ x }  \right) \\
&= -2 \left( e \circ \overline{ x } \right)  ^T W \left( \delta \circ \overline{ x }  \right) + 2 \left( \delta \circ \overline{ x }  \right) ^T W \left( \delta \circ \overline{ x }  \right) \\
&\overset{ \ref{lem:calcrule}  }{ =}  -2 \delta ^T \left( W \circ \overline{ x } \overline{ x } ^T \right)  e + 2 \delta ^T \left( W \circ \overline{ x } \overline{ x } ^T  \right) \delta \\
&= -2 \delta ^T \text{diag} \left( ( W \circ \overline{ x } \overline{ x } ^T ) e  \right) \delta + 2 \delta ^T \left( W \circ \overline{ x } \overline{ x } ^T  \right) \delta 
\overset{ \text{def } M ( \overline{ x }  )   }{=}  2 \delta M ( \overline{ x }  ) \delta
\end{align*} 
In the second to last equality we used $ \delta ^T v = \delta ^T \text{diag} \left( v \right) \delta $, which follows immediately from
\begin{align*}
\left(  \text{diag} (v) \delta \right) _{ i } = 
\begin{cases}
0 & \text{if } \delta _{ i } = 0 \\
v_i & \text{if } \delta _{ i } = 1
\end{cases}
\quad \text{ for all } i \in \left[ 1:n \right].
\end{align*} 
So far we have shown $ 0 \leq \delta ^T M ( \overline{ x }  ) \delta  $, however for nonnegative summability we need to argue $ 0 \leq y ^T M ( \overline{ x }
) y  $ for all $ y \in \left\{ 0,1 \right\} ^{ n }  $.\\
\textbf{Claim}: For any $ y \in \left\{ 0,1 \right\} ^{ n } $ there exists a $ x \in \left\{ -1,1 \right\} ^{ n }  $ such that $ \delta = y $. \\
First recal that $ \delta \in \left\{ 0,1 \right\} ^{ n } $ depends on $ \overline{ x }  $, which is fixed, and $ x $, which is a variable. 
To clarify this dependence we denote $ \delta $ as $ \delta _{ \overline{ x }  } \left( x \right)  $.
As $ x \in \left\{ -1,1 \right\} ^{ n } $ can be chosen arbitrarily, we choose
\begin{align*}
x_i = \begin{cases}
\overline{ x_i }  & \text{if } y_i=0 \\
- \overline{ x_i } & \text{if } y_i=1
\end{cases}
\end{align*} 
This choice directly implies $ \delta _{ \overline{ x }  } \left( x \right) = y $.
The claim shows that $ M ( \overline{ x }  )  $ is nonnegatively summable.
Now we want to show that $ \overline{ x }  $ is minimum cut if and only if $ -  M( \overline{ x } ) $ is nonnegatively summable. 
We do this in the same way as above only outlining the slight changes. Let $ q $ be the quadratic function we defined above, $ \overline{ x } \in \left\{ -1,1 \right\} ^{ n }
$ a minimum cut and $ x \in \left\{ -1,1 \right\} ^{ n }  $ arbitrary. Thus we have $ 0 \geq  q(x) - q( \overline{ x } ) $ and the same compuations as above yield $ 0 \geq
\delta ^T M ( \overline{ x } ) \delta \Leftrightarrow 0 \leq \delta ^T  \left( -M( \overline{ x } ) \right) \delta  $. The nonnegative summability of $ - M(
\overline{ x } ) $ follows in the same manner as above.
\end{proof}
\begin{rem}
\begin{enumerate}
\item 
Let $ \overline{ x } \in \left\{ -1,1 \right\} ^{ n }  $.
Then the map
\begin{align*}
\delta _{ \overline{ x }  }  : \left\{ -1,1 \right\} ^{ n }  \to \left\{ 0,1 \right\} ^{ n }  , \ x \mapsto \delta _{ \overline{ x }  } ( x ), \text{ where }   \delta _{ \overline{ x
} } (x)  _{ i } = \begin{cases}
0 & \text{ if } x_i = \overline{ x_i } \\
1 & \text{ if } x_i \neq \overline{ x_i }  
\end{cases}
\text{ for all } i \in \left[ 1:n \right]   
\end{align*} 
is a bijection. \\
In the previous proof we have shown that $ \delta _{ \overline{ x }  }  $ is surjective and as $ \left\{ -1,1 \right\} ^{ n } , \left\{ 0,1 \right\} ^{ n }  $ are finite sets
of the same cardinality we have bijectivity.
\item The matrix $ M ( \overline{ x } ) $ is symmetric for every $ \overline{ x } \in \left\{ 0,1 \right\} ^{ n }  $: \\
The matrix $ \text{diag} \left( ( W \circ \left( \overline{ x } \overline{ x } ^T  \right) e  \right)  $ is symmetric as it is a diagonal matrix.
As $ W $ is symmetric, we have for every $ i,j \in \left[ 1:n \right]  $:
\begin{align*}
\left( W \circ \overline{ x } \overline{ x } ^T	 \right) _{ ij } = w _{ ij } \overline{ x } _{ i } \overline{ x } _{ j } = w _{ ji } \overline{ x } _{ j } \overline{ x } _{ i
} = \left( W \circ \overline{ x } \overline{ x } ^T  \right) _{ ji } 
\end{align*} 
This shows that $ \left( W \circ \overline{ x } \overline{ x } ^T  \right)  $ is symmetric and, as a difference of symmetric matrices, $ M ( \overline{ x } )  $ is a symmetric
matrix.
\end{enumerate}
\end{rem} 
Comparing the Hessian of $ f $ as given in \ref{def:Hessian} to \ref{def:M(x)}, we see that the two look similar. 
The difference is that $ \cos( T(\theta) )  $ in \ref{def:Hessian} is replaced by $ x  x  ^T  $ in \ref{def:M(x)}. Comparing the entries we see that $
\cos( \theta_i - \theta_j ) = x_i x_j  $ for all $ i,j \in \left[ 1:n \right]  $ is a sufficent condition for $ H(\theta) = M(x) $.
This observation leads to the next theorem providing a classification of angular representations of cuts as stationary points of the function 
$ f $.
\begin{thm}
\label{thm:classificationcuts} 
Let $ \overline{ \theta }  $ be a angular representation of a cut, or simply a cut, and let $ \overline{ x } \equiv x( \overline{ \theta } ) $ be the associated binary cut. If $ \overline{ \theta }  $ is a local minimum
(respectively, local maximum) of $ f (\theta) $, then $ \overline{ x }  $ is a maximum (respectively, minimum) cut). 
Consequently, if $ \overline{ x }  $ is neither a maximum cut nor a minimum cut, then $ \overline{ \theta }  $ must be a saddle point of $ f $.
\end{thm} 
\begin{proof}
First recall that every angular representation of a cut is a critical point of $ f $ and that 
$ \cos( \overline{ \theta_i } - \overline{ \theta_j }  ) = \overline{ x }_i \overline{ x }_j  $ for all $ i,j \in \left[ 1:n \right]  $.
Therefore $ H( \overline{ \theta } ) = M( \overline{ x } ) $ holds.
Let $ \overline{ \theta }  $ be a local minimum of $ f $.
As $ f $ is smooth $ H ( \overline{ \theta } ) $ is positive semidefinite and in particular nonnegatively summable. By $ H ( \overline{ x } ) = M ( \overline{ x } ) $ we have
that $ M ( \overline{ x } ) $ is nonnegatively summable.
Now let $ \overline{ \theta }  $ be a local maximum of $ f $. Then $ H ( \overline{ \theta } ) $ is negative semidefinite, that means that $ - H ( \overline{ \theta } ) $ is
nonnegatively summbale. Therefore $ - H ( \overline{ \theta } ) = - M ( \overline{ x } ) $ implies that $ - M ( \overline{ x } ) $ is nonnegatively summable.
Lastly let $ \overline{ x }  $ be neither a maximum cut nor a minimum cut.
By Lemma \ref{lem:M(x)nonnegsum} we know that neither $ M ( \overline{ x } ) $ nor $ - M ( \overline{ x } ) $ is nonnegatively summable. 
By $ M ( \overline{  x } ) = H ( \overline{ \theta } ) $ we have that $ H ( \overline{ \theta } )  $ is neither positive semidefinite nor negative semidefinite. By the
necessary condition for a local extremum we conclude that $ \overline{ \theta }  $ is neither a local minimum nor a local maximum. As $ \overline{ \theta }  $ is a critical
point of $ f $, it must be a sattle point.
\end{proof}
Although all cuts are stationary points of $ f $ (by \ref{prop:cutsArestationaryPoints}) Theorem \ref{thm:classificationcuts} shows that only the maximum points can be local minima of $ f $, see \cite[p. 508]{Burer2002}.
\begin{exa}
The converse of the two implications in the above theorem do not hold. We will give an example showing that the angular representation of a maximum cut is not necessarily a
local minimum of $ f $. \\
Consider the complete graph with three vertices, where every edge has weight $ 1 $.
Then the weight of a maximum cut is $ 2 $ and is achieved by 
$ \overline{ x } = \begin{pmatrix}
1 & -1 & -1
\end{pmatrix}   $.
We then have 
\begin{align*}
W \circ \overline{ x } \overline{ x } ^T = \begin{pmatrix}
1 & -1 & -1 \\
-1 &1 & 1 \\
-1 & 1 & 1
\end{pmatrix} 
\text{ and } 
\left( W \circ \overline{ x } \overline{ x } ^T \right)  e = \begin{pmatrix}
-1 \\
1 \\
1
\end{pmatrix} 
\end{align*} 
Using \ref{def:M(x)} we get
\begin{align*}
M( \overline{ x } ) = 
\begin{pmatrix}
1 & -1 & -1 \\
-1 &1 & 1 \\
-1 & 1 & 1
\end{pmatrix} 
- 
\begin{pmatrix}
-1 & 0 & 0 \\
0 & 1 & 0 \\
0 & 0 & 1
\end{pmatrix} 
= \begin{pmatrix}
2 & -1 & -1 \\
-1 & 0 & 1 \\
-1 & 1 & 0 \\
\end{pmatrix} 
\end{align*} 
Straight forward computation shows nonnegative summability, namely $ u ^T M ( \overline{ x } ) u \geq 0  $ for all $ u \in \left\{ 0,1 \right\} ^{ n }  $.
However, $ M ( \overline{ x } )  $ is not positive semidefinite, since
\begin{align*}
\begin{pmatrix}
1 & 0 & 2
\end{pmatrix} 
M ( \overline{ x } ) 
\begin{pmatrix}
1 \\ 
0 \\
2
\end{pmatrix} 
= 
\begin{pmatrix}
1 & 0 & 2
\end{pmatrix} 
\begin{pmatrix}
0 \\
1 \\ 
-1
\end{pmatrix} 
= -2 < 0
\end{align*} 
Using $ H ( \theta ( \overline{ x } ) )= M ( \overline{ x } ) $, we know that $ H ( \theta ( \overline{ x } ) ) $ is not postive semidefinite. Thus $ \theta ( \overline{ x } )
$ is not a local minimum of $ f $.
\end{exa} 
However there special classes of instances, where for every maximum cut $ \overline{ x }  $, the angular representation $ \theta ( \overline{ x } ) $ is a local minimum. 
The following Proposition gives a class for which this holds:
\begin{prop}
For a bipartite graph with nonnegative edges, the global minimum value of $ f $ is attained by a maximum cut.
\end{prop} 
\begin{proof}
We denote the bipartite graph $ G = \left( A,B,E \right)  $.
As all the edge weights are nonnegative, the cut $ ( A, B) $, that cuts through all edges is a maximum cut.
For that cut we have $ \cos( \theta _{ i } - \theta _j ) = -1 $ for all $ \left\{ i,j \right\} \in E $.
For that cut $ f $ evaluates to $ - \frac{ 1 }{ 2 } e ^T W e  $.
This value must be a global minimum of $ f $ as all the entries of $ W $ are nonnegative.
\end{proof}
For instances, where a maximum cut $ \overline{ x }  $ corresponds to a local minimum $ \theta ( \overline{ x } ) $ of $ f $, the optimality of $ x $ can be
checked in polynomial time. That is because $ \theta ( \overline{ x } ) $ local minimum implies $ H ( \theta ( \overline{ x } ) = M ( \overline{ x } ) $ postive
semidefinite. This in turn implies $ M ( \overline{ x } )  $ nonnegatively summable, which shows $ \overline{ x }  $ maximum cut by \ref{lem:M(x)nonnegsum}. 
A detailed proof showing that we can decide whether a symmetric matrix (- $ M ( \overline{ x } ) $ is symmetric-) is positive semidefinite in polynomial time is provided in \cite{Korte2018} Theorem 16.8.

Theorem \ref{thm:classificationcuts} directly implies the following Corollary, which plays an importrole for the heuristic we will develop in the next section.
\begin{cor}
Let $ x \in \left\{ 0,1 \right\} ^{ n } $ be a nonmaximum cut. Then $ \theta ( x )  $ can not be a local minimum.
\end{cor} 
\begin{proof}
%Assume $ \theta ( x) $ is a local minimum of $ f $. Then $ x $ is maximum cut, by \ref{thm:classificationcuts}. 
The statement is the contraposition of the first statement in theorem \ref{thm:classificationcuts} 
\end{proof}
Therefore, as written in \cite[p.509]{Burer2002}, a good minimisation algorithm would not be attracted to stationary points which are not local minima. 
We construct such an alogrithm in section \ref{burerHeuristic}.
\newpage
\section{A heuristic algorithm for \MCP} 
\label{burerHeuristic} 
In this section we will describe the heuristic described by \cite{Burer2002}.
Again we assume nonnegative edge weights, in order to mirror the analysis done in \ref{sec:GoemansWilliamson} 

We begin by minimising $ f $, i.e. finding a solution $ \theta $ to \ref{def:minf}.
Note that $ \theta $ is not necessarily an angular representation of a cut.
Therefore we need to develop a method associating a cut $ x \in \left\{ -1,1 \right\} ^{ n }  $ to $ \theta $.
Using polar coordinates we can associate the entries of $ \theta $ to points on the unit circle.
Furthermore we can assume $ \theta_i \in \left[ 0, 2\pi \right)  $ for all $ i \in \left[ 1:n \right]  $.
As in section \ref{sec:GoemansWilliamson} we can generate a cut by cutting the unit circle into two halves. For any angle $ \alpha \in \left[ 0, \pi \right)  $ we can set
\begin{align}
\label{eq:assigncut} 
x_i = \begin{cases}
1 & \text{if } \theta_i \in \left[ \alpha , \alpha + \pi \right) \\
-1 & \text{otherwise}
\end{cases}
\end{align} 
generating a cut.
The weight of the obtained cut is given by
\begin{align*}
\gamma (x) = \frac{ 1 }{ 2 } \sum_{ i>j   }^{  } w _{ ij } \left( 1 - x_i x_j \right) 
\end{align*} 
\begin{rem}
\begin{enumerate}
\item 
Let $ \theta \in \left[ 0, 2 \pi  \right) ^{ n }  $ and $ \alpha \in \left[ 0,\pi \right)  $.
Let $  x ^{ \alpha }  $ denote the cut generated by $ \alpha $ and $  x ^{ \alpha + \pi  }  $ the cut generated by $ \alpha + \pi $,
we then have $ x  ^{ \alpha + \pi } = - x  ^{ \alpha }  $. 
For $ \alpha + \pi  $ assignement \ref{eq:assigncut} is to be understood as:
\begin{align*}
x_i = \begin{cases}
1 & \text{if } \theta_i \in \left[ 0, \alpha \right) \cup \left[ \alpha + \pi , 2 \pi  \right) \\
-1 & \text{otherwise} 
\end{cases}
\end{align*} 
By observing $ \left[ 0 , 2 \pi  \right) = \left[ \alpha, \alpha + \pi  \right) \cupdot \left(  \left[ 0, \alpha
\right) \cup \left[ \alpha + \pi , 2 \pi  \right)  \right) $ we deduce $ - x ^{ \alpha } = x ^{ \alpha + \pi  }  $.
\item 
Choosing the halfopen interval $ \left[ \alpha, \alpha + \pi  \right) $ ensures, that diametrially opposed points are seperated, i.e. the corresponding vertices are on
diferent sides of the cut.
\textcolor{red}{This is a positive feature:
We denote  $ x ^{ \text{GW}  }  $ .... and $ x ^{ \text{B}  }  $ the assignement obtained by \ref{eq:assigncut} \\
We have $ x ^{ \text{GW}  } _{ i } \neq x ^{ \text{B}  } _{ i } $ if and only if $ \theta_i = \alpha + \pi  $
}
\textcolor{red}{
Because of this we will allow ourselves the slight inaccuracy and claim that \ref{eq:assigncut} finds all the cuts. 
It can be understood as finds all the best possible Goemans-Williamson-type cuts. 
}
\end{enumerate}
\end{rem} 
As we are given a set of points on the unit circle, it is easy to examine all the possible cuts obtained by assignement \ref{eq:assigncut}. 
That is because the unit circle is one dimensional and we are able to traverse all Goemans-Williamson-type cuts by controlling only one variable.
Thus a big difference to the algorithm in \ref{sec:GoemansWilliamson} is that we have a deterministic way of obtaining the best possible \textcolor{red}{Goemans-Williamson-type
cut} associated to a given $ \theta $.
\todo[inline]{Actually we do not necessarily traverse all possible Goemans-Williamson-type cut as, diametrially opposed points are always separated. \\
However we should be able to argue that, we still find the best one, reasoning something like this: \\ 
The vectors are diametrially opposed for a good reason, as the points solve the vector program, and they could have been chosen differently. This argument does not feel
watertight anymore, as we do not solve the vector program optimally, but only obtain a local min. \\
Maybe this is good enough, as we may still be able to argue, that for a given set of points the diamatral opposition is optimal, if it weren't slightly moving one point could
improve the cut. Contradiction to local minimum of $ f $.
}
Before we describe the algorithm in detail, we will make two assumptions without loss of generality.
First of all we note that by the $ 2\pi $-periodicity in each variable of $ f $, see \ref{prop:propertiesf} , we can assume $ \theta_i \in \left[ 0,2\pi \right)  $ for all $ i
\in \left[ 1:n \right]  $.
Furthermore we assume $ \theta_1 \leq \theta_2 \leq \dots \leq \theta_n  $.
\begin{algorithm}
\caption{Procedure-Cut}
\label{alg:ProcedureCUT} 
\begin{algorithmic}[1]
\Require Undirected weighted graph $ G $
\Ensure A cut $ x^* $
\State Let $ x^* $ trivial cut, i.e. all vertices on one side
\State Let $ \alpha =0, \Gamma = -\infty , i =1 $. Let $ j $ be the smallest index such that $ \theta_j > \pi $ if there is one; otherwise set $ j = n+1 $. Set $ \theta _{ n+1
} = 2\pi $.
\While{ $ \alpha < \pi $}
\State Generate cut $ x $ by \ref{eq:assigncut} and compute $ \gamma (x) $
\label{gencut} 
\State If $ \gamma (x) > \Gamma $, then let $ \Gamma = \gamma (x) $ and $ x^* = x $.
\If{ $ \theta_i \leq \theta_j - \pi  $ }
  \State Let $ \alpha = \theta_i $ and increment $ i $ by 1.
\Else 
\State Let $ \alpha = \theta_j - \pi $ and increment $ j $ by 1.
\EndIf
\EndWhile
\State \Return $ x ^{ * }  $
\end{algorithmic}
\end{algorithm}
\begin{rem}
The assumption $ \theta_1 \leq \theta_2 \leq \dots \leq \theta_n  $ is not a real constraint, we just have to keep track the original indices, as they determine the cut.
%Hence let us consider the tuples $ \left( \theta_i,i \right)  $ for $ i \in \left[ 1:n \right]  $ and sort them by the values of the first component in increasing order.
%Now we can apply algorithm \ref{alg:ProcedureCUT}, with the only difference that the cut in 
%The points $ \theta_1, \dots, \theta_n $ can be sorted by their angle in increasing order, denoted as $ \widehat{ \theta } _{ 1 } , \dots, \widehat{ \theta } _{ n }  $.
%We denote the bijective function associating 
Let $ \sigma $ be a permutation such that $ \theta _{ \sigma (1) } \leq \theta _{ \sigma (2) } \leq \dots \leq \theta _{ \sigma (n) }  $. 
We can now apply algorithm \ref{alg:ProcedureCUT}, with the only slight change that the cut in line \algref{alg:ProcedureCUT}{gencut} needs to be generated in the following manner:
\begin{align*}
x_i = \begin{cases}
1 & \text{if } \theta _{ \sigma (i) } \in \left[ \alpha, \alpha + \pi  \right)  \\
-1 & \text{otherwise} 
\end{cases}
\end{align*} 
\end{rem} 
\begin{lem}
\label{lem:algGeneratesAllgwtc} 
The algorithm \ref{alg:ProcedureCUT} returns a Goemans-Williamson-type cut of maximum weight ( with respect to the input) 
\end{lem} 
\begin{rem}
Intuitively, this can be argued by slightly rotating the dividing line (through the origin) in the mathematically negative sense. 
Just enough so that the points with angle $ \alpha + \pi  $ are separated from the points with angle $ \alpha $.
But not so far, that any other points change their assignment.
In the following we will make this precise.
\end{rem} 
\begin{proof}[Proof of Lemma \ref{lem:algGeneratesAllgwtc}]
To show that the returned cut is a Goemans-Williamson-type cut, it suffices to show that for every $ \alpha \in \left[ 0, \pi  \right) $ assignement \ref{eq:assigncut} can be
achieved by a Goemans-Williamson-type cut. 
We begin by showing that th
Assignement \ref{eq:assigncut} states: $ x_i = 1 $ if $ \theta_i \in \left[ \alpha, \alpha + \pi  \right) $. As this is a half open interval, we need to argue that assignment \ref{eq:assigncut} can
be achieved by a Goemans-Williamson-type cut: \\
Let $ P = \left\{ i \in \left[ 1:n \right]  \mid \theta_i \in \left[ \alpha, \alpha + \pi  \right) \right\}   $, denote the indices that are assigned to $ x_i = 1 $. The
indices that are assigned to $ x_i = -1 $ are denoted by $ N := \left[ 1:n \right] \setminus P $.
We denote the smallest angle of points with indices in $ N $ to the point 
$ \begin{pmatrix}
\cos( \alpha   ) & \sin( \alpha   ) ) 
\end{pmatrix}  ^{ T } $, 
by 
\begin{align*}
 \varepsilon _{ N }  := \min_{i \in N} \arccos \left( 
\begin{pmatrix}
\cos( \theta_i ) \\
\sin( \theta_i )  
 \end{pmatrix} ^T 
\begin{pmatrix}
\cos( \alpha   ) \\
\sin( \alpha  )  
 \end{pmatrix}  \right) 
\end{align*} 
Likewise, we denote the smallest angle of points with indices in $ P $ to the other point 
$ \begin{pmatrix}
\cos( \alpha + \pi  ) & \sin( \alpha + \pi  ) ) 
\end{pmatrix}  ^T $
, by 
\begin{align*}
 \varepsilon _{ P }  := \min_{i \in P} \arccos \left( 
\begin{pmatrix}
\cos( \theta_i ) \\
\sin( \theta_i )  
 \end{pmatrix} ^T 
\begin{pmatrix}
\cos( \alpha + \pi  ) \\
\sin( \alpha + \pi )  
 \end{pmatrix}   \right) 
\end{align*} 
We have $ \varepsilon _{ N } , \varepsilon _{ P } >0 $, since $ \alpha \notin \left\{ \theta_i \mid i \in N \right\}  $ and $ \alpha + \pi \notin \left\{ \theta_i \mid i \in P
\right\}  $.
%More precisely, since the $ \theta_i $'s are a discrete set on the unit circle there exists a $ \varepsilon := \frac{ 1 }{ 2 } \min_{} \left\{ \alpha - \min_{ i \in N} \theta_i ,
%\alpha + \pi - \max_{ i \in P} \theta_i \right\} > 0 $.
Now we can choose $ \varepsilon \in \left( 0, \min_{} \left\{ \varepsilon _{ P } , \varepsilon _{ N }  \right\} \right) $ and consider the Goemans-Williamson-type cut generated
by $ \begin{pmatrix}
\cos( \alpha + \frac{ \pi  }{ 2 } - \varepsilon  ) &
\sin( \alpha + \frac{ \pi  }{ 2 } - \varepsilon  ) 
\end{pmatrix}  $ 
This leads to the cut
\begin{align*}
x ^{ \text{GW}  } _i = \begin{cases}
1 & \text{if } \theta_i \in \left[ \alpha - \varepsilon , \alpha + \pi - \varepsilon  \right] \\
-1 & \text{otherwise} 
\end{cases}
\end{align*} 
In case of $ \alpha - \varepsilon < 0  $, we have $ \alpha - \varepsilon > - \pi  $, the interval $ \left[ \alpha - \varepsilon , \alpha + \pi - \varepsilon  \right] $ is to
be understood as $ \left[ 0, \alpha - \varepsilon + \pi  \right] \cup \left[ 2 \pi + \alpha -
\varepsilon , 2 \pi  \right)  $.
We need to verify $ x = x ^{ \text{GW}  }  $. As the variable is binary it suffices to show $ x_i = 1 $ implies $ x ^{ \text{GW}  } _{ i } = 1 $. Let $ x_i = 1 $, 
then by construction of $ \varepsilon _{ P }  $ we have $ \theta_i \in \left[ \alpha, \alpha + \pi - \varepsilon _{ P }  \right]  $. By construction of $ \varepsilon  $, we
have $ \theta_i \in \left[ \alpha - \varepsilon , \alpha + \pi - \varepsilon  \right]  $ and thus $ x ^{ \text{GW}  } _{ i } = 1 $.
Similarly, for $ x_i = -1 $, we have $ \theta_i \in \left[ \alpha + \pi , 2 \pi \right) \cup \left[ 0 , \alpha - \varepsilon _{ N }  \right]  $ 
\improvement{This interval may not make sense , i.e. if $ \alpha - \varepsilon _{ N } < 0   $, in that case the interval is to be understood as $ \left[ \alpha + \pi , 2 \pi +  \alpha - \varepsilon _{ N }  \right]  $}
. By $ 0 < \varepsilon < \varepsilon _{ N }  $ we in particular have $ \theta_i \in \left( \alpha + \pi - \varepsilon , 2 \pi  \right) \cup \left[ 0, \alpha - \varepsilon
\right)  $, which shows $ x ^{ \text{GW}  } _{ i } = -1 $.
\todo[inline]{So far we have shown that the Goemans-Williamson-type cut between two .. cuts eval to the same \\
This doesnt show that all Goemans-Williamson-type cuts can be gen by \ref{eq:assigncut}. The ones missing are the ones where $ \theta_i = \alpha, \theta _{ j } = \alpha + \pi
$ are in the same set. This is not a problem if all weigths are nonnegative, also the two being diametrially opposed is a better cut (as the other assignements remain
unchanged). However if that happens to be a negative weight the cut we don't find might be better. This is import with regard to the results as we have instances, with neg
weights. Also remember that GW algo assumes nonnegative weights. }
\end{proof}

\textcolor{red}{As our rank-two relaxation has the form of the Goemans and Williamson relaxation with $ n=2 $, essentially the same analysis applies.}
\textcolor{red}{Let us quickly go over the changes: \\
As $ f $ is nonconvex, we can not expect to find the global minimum, only a local minimum.
This destroys the performance guarantee of the algorithm.
Let us show how to appropriately change this: \\
The local minium of $ f $, say $ \theta \in \mathbb{R} ^{ n } $ represents the points 
$
v_i =
\begin{pmatrix}
\cos( \theta_i ) &
\sin( \theta_i ) 
\end{pmatrix} ^{ T }  
$ for $ i \in \left[ 1:n \right]  $.
As all these points are on the unit circle, $ v_1 , \dots, v_n $ are a feasible solution to \ref{def:relaxedbqp} and their value of the objective function is given by ...
\todo[inline]{CONTINUE}.
This is closely linked with the fact that we do not solve the \SDP.
On the other hand, since we fixed $ n = 2 $, we have a methodical and efficient way to find the cut values of all possible Goemans-Williamson-type cut, see
\ref{lem:algGeneratesAllgwtc}.
}
\todo[inline, caption={2do}]{We are going to go into the changes: \\
\begin{enumerate}
\item The cut we obtain is not probabilistic but optimal, for the distribution of points
\item We did not solve the SDP so we arent at the global optimum
\end{enumerate}
}
Therefore the cut value generated by \ref{alg:ProcedureCUT} is at least 0.878 times the \textcolor{red}{relaxed cut value $ \psi (\theta) $ as defined in
\ref{eq:psiRelaxedCut} }, written in a formuala:
\improvement{Maybe put that in lemma}
\begin{align}
\label{eq:perfguarantee} 
\gamma (x^*) \geq 0.878 \psi (\theta) 
\end{align} 
This is not a performance guarantee, as we can not guarantee that $ \psi ( \theta)  $ is an upper bound on the maximum cut value.
\textcolor{red}{Recall, that in Goemans-Williamson algorithm, the guarantee comes from solving the \textcolor{red}{vector program}}
\improvement{Vector Program not yet introduced} 
The next Lemma can be interpreted as a weak performance guarantee
\begin{lem}
Let $ x ^{*} _{ a } $ and $ x ^{*} _{ b } $ be two cuts generated by algorithm \ref{alg:ProcedureCUT} from $ \theta_a $ and $ \theta_b $ respectively.
If $ \gamma ( x ^{ * } _{ a } ) \leq \psi ( \theta) $ and $ \psi ( \theta _{ b } ) > \psi ( \theta_a)/0.878 $, then $ x ^{ * } _{ b }  $ has a higher cut value than $ x ^{ * }
_{ a }  $.
\end{lem} 
\begin{proof}
Using the assumptions we confirm
\begin{align*}
\gamma ( x ^{ * } _{ b } ) \overset{ \ref{eq:perfguarantee}  }{ \geq  } 0.878 \cdot \psi ( \theta_b) > \psi ( \theta_a) \geq \gamma ( x ^{ * } _{ a } ) 
\end{align*} 
\end{proof}

\textcolor{red}{ The next paragraph, describes the idea of \ref{alg:improvedCut} }
Minimize the function $ f $ and obtain a minimiser called $ \theta^1 $. 
We can now start \ref{alg:ProcedureCUT} and obtain a best possible cut $ x^1 $ associated with $ \theta^1 $.
At this point we can return the cut $ x^1 $.
\textcolor{red}{If $ \theta^1 $ happens to be an angular representation of a cut we know that $ x^1 $ is a maximum cut, by theorem \ref{thm:classificationcuts}.}
Otherwise, we can continue to try and improve the cut value, if we are willing to spend the computational resources. 
By theorem \ref{thm:classificationcuts} we know that the angular representation of $ x^1 $, denoted as $ \theta ( x^1)  $, is a stationary point of $ f $. 
\textcolor{red}{As we use a gradient descend,} restarting the minimisation from a stationary point is of no use.
Thus we restart the minimisation of $ f $ from a slight perturbation of $ \theta (x^1) $, in hopes of finding another local minimum $ \theta^2 \neq \theta^1  $ from which we
hope to obtain a cut $ x^2 $ with a higher cut value than $ x^1 $, i.e. $ \gamma (x^2) > \gamma (x^1)  $.
\textcolor{red}{ In case $ \gamma (x^2) > \gamma (x^1) $ is achieved we deem our attempt succesful, and unsuccesful otherwise. }
We can set the termination condition to be N unsuccesful successive attempts of improving the value of the cut.
\begin{algorithm}
\caption{\ImprovedCut}
\todo[inline]{Find a better name}
\label{alg:improvedCut} 
\begin{algorithmic}[1]
\Require Undirected weighted graph $ G $
\Ensure A cut $ x^* $
\Procedure{\ImprovedCut}{input $ N, \theta^0 $}
\State Given $ \theta^0 \in \mathbb{R} ^{ n }  $ and integer $ N \geq 0  $, let $ k=0 $ and $ \Gamma = - \infty  $
\While{ $ k \leq N  $}
\State Starting from $ \theta^0 $, minimize $ f $ to get $ \theta  $
\State Compute a best cut $ x $ associated with $ \theta $ by \ref{alg:ProcedureCUT} 
\If{$ \gamma (x) > \Gamma $}
\State Let $ \Gamma = \gamma (x), x^* = x $ and $ k=0 $
\Else
\State $ k = k+1 $
\EndIf
\State Set $ \theta^0 $ to a random perturbation of the angular representation of $ x $.
\EndWhile
\State \Return $ x ^{ * }  $
\EndProcedure
\end{algorithmic}
\end{algorithm}

In \ref{alg:improvedCut} we have $ \theta^0 $ as input, which is the initial seed for the minimisatio of $ f $. We can try and improve our chances of finding a high quality
heuristic solution by increasing running \ref{alg:improvedCut} multiple times, in other words increase the number of seeds. The number of seeds, will be given by the input M.
\improvement{Write how increasing M and N usually leads to better solutions but the runtime can nnot be vorhergesagt. Also there can be lucky punches, say start at the angular
representation of a cut.}

The algorithm \ref{alg:improvedCut}, can also be interpreted in the following way.
\improvement{We assume that we do not find the optimum cut.}
From a seed we start minimising $ f $. By means of Goemans-Williamson-type cuts, we look for nearby sattlepoints. 
\textcolor{red}{Here it is important to note, that nearby is to be understood by the cuts that we can generate from the minimum. The so found sattlepoint is not necessarily
the closest sattlepoint in the euclidean metric.}
The sattlepoint is chosen by the associated cut with the highest weight.
From a slight perturbation of that point we search for a local minimum, which has a sattlepoint with lower $ f $-value in its neighborhood.
\begin{algorithm}
\caption{BurerStub}
\label{alg:BurerStub} 
\todo[inline]{Find better name?}
\begin{algorithmic}[1]
\Require Undirected weighted graph $ G $
\Ensure A cut $ x^* $
\State Let $ x^* $ trivial cut, i.e. all vertices on one side
\For{ $ i \gets 0,n $}
  \State Generate random $ \theta^0 \in \mathbb{R} ^{ n } $ 
  \State $ x \gets \Call{\ImprovedCut} $
  \If{ $ \gamma(x) > \gamma(x^*) $} 
    \State $ x^* \gets x $
  \EndIf
\EndFor
\State \Return $ x^* $
\end{algorithmic}
\end{algorithm}
\todo[inline]{reference of algorithms doesnt work! fix in whole file!}
\newpage
\section{Solving BQP with this heuristic} 
\label{BQP2MC} 
Closely following \cite[]{Mallach2021} we are going to give a cursory explanation, that binary quadratic problems of the form:
Let $ Q \in \mathbb{R} ^{ n \times n }  $
\begin{mini}
{}{ x^TQx }{}{}
\label{def:01bqp}
\addConstraint{}{   x_i   \in \left\{ 0,1 \right\}       }{ \forall i \in  [1:n]}
\end{mini}
can be solved by solving the \MCP on a suitable graph with appropriately chosen edge weights. 
\improvement{Maybe begin by lemma showing that if there are no diagonal entries it is fairly simple}
\todo[inline]{This should orient itself to Mallachs Paper. Something along the lines of This is a BQP we are interested in solving, this is the way to transform it to a MCP}
The goal of this \textcolor{red}{section} \improvement{choose right word if not section, will see later} is to lay out the relationship between the \textcolor{red}{BQP} and
\textcolor{red}{MC}. 
\improvement{We have done a special case of this in \ref{lem:01}, where $ Q $ is a strict upper diagonal matrix. However the following construction is much more general.}

To this end we formulate the \MCP as an integer linear programming problem:
\todo[inline]{def opti}

\begin{mini}
{}{  \sum_{ e \in E  }^{  } w_e z_e }{}{}
\label{def:mcpILP} 
\addConstraint{  \sum_{ e \in F  }^{  } z _{ e } - \sum_{ e \in C \setminus F  }^{  }  }{  \leq \left| F  \right| -1  }{ \quad  F \subseteq C \subseteq E  \text{ subject to } C \text{ cycle and } \left| F \right| \text{ odd} }
\addConstraint{  0 \leq z_e    }{ \leq 1}{ \quad \forall e \in E}
\addConstraint{   z_e    }{ \in  \mathbb{Z}  }{ \quad \forall e \in E}
\end{mini}

\improvement{give the intuition for odd cycle constraints}
An optimal solution to \ref{def:mcpILP} gives us the following maximum cut $ \left\{ \left\{ i,j \right\}  \mid z _{ ij } = 1 \right\}  $

The cut polytope of a graph $ G $ can thus be denoted as 
\begin{align*}
P ^{ G } _{ \text{CUT}  } := \left\{ z \in \left[ 0,1 \right] ^{ E }   \mid  z \text{ is a feasible solution to } \ref{def:mcpILP} \right\} 
\end{align*} 
%To get a grasp of the problem we start by treating the special case where all the diagonal entries of $ Q \in \mathbb{R} ^{ n \times n}  $ are zero, i.e. $ \text{diag} (Q) = 0 $.
%In the rest of this paragraph we will motivate the choice of the corresponding graph.
%In this special case the construct
%Then $ Q $ induces a directed graph without loops, where the edge $ (i,j) $ has weight $ q _{ ij }  $. 
%As there are no loops we can simply considered the underlying undirected graph, where we assign the weight $ q _{ ij } + q _{ ji }  $ to $ \left\{ i,j \right\} $.
%With this construction we may have introduced edges of weight zero, which we can simply delete.
%
%\begin{lem}
%Let $ Q \in \mathbb{R} ^{ n \times n }  $ such that $ \text{diag} (Q) = 0 $ and consider $ G = ( V,E) $ with $ V := \left[ 1:n \right]  $ and $ E := \left\{ \left\{ i,j
%\right\}  \mid i,j \in \left[ 1:n \right] : q _{ ij } + q _{ ji } \neq 0  \right\}  $. 
%The edge $ \left\{ i,j \right\} \in E  $ is assigned the weight $ q _{ ij } + q _{ ji }  $.
%Then solving \ref{def:01bqp} is \textcolor{red}{equivalent} to solving the \MCP on $ G $.
%\end{lem} 
%\begin{proof}
%The idea is to notice that $ q _{ ij }  $ counts towards the value of the objective function in \ref{def:01bqp} if and only if $ x_i x_j = 1$.
%As $ x_i x_j $ evaluates to either $ 0 $ or $ 1 $ we get the same objective function value by first adding all the entries of $ Q $ and then subtracting the entries $ q _{ ij
%}  $for which $ x_i x_j $ evaluates to $ 0 $. This is what is done in the computation below:
%\begin{align*}
%x^T Q x = \sum_{ i,j = 1 }^{  } q _{ ij } x_i x_j = \sum_{ i,j = 1 }^{ n } q _{ ij } - \sum_{ i,j = 1 }^{ n } q _{ ij } (1 - x_i x_j)
%\end{align*} 
%Noticing that $ \sum_{ i,j = 1 }^{ n } q _{ ij }  $ is constant and using the bijection 
%\begin{align*}
%f : \left\{ 0,1 \right\} ^{ n }	 \to  \left\{ -1,1 \right\} ^{ n } , \  x \mapsto 2x-e
%\end{align*} 
%\todo[inline]{Proof not finished!}
%\end{proof}

Without proof we state the following theorem proven in \cite{DeSimone1990}.
\begin{thm}
\label{thm:simone} 
Let $ Q \in \mathbb{R} ^{ n \times n }  $ and consider $ G = ( V,E ) $ with $ V := \left[ 1 : n \right] $ and $ E := \left\{ \left\{ i,j \right\}  \mid i,j \in \left[ 1:n
\right] : q _{ ij } + q _{ ji } \neq 0   \right\}  $. Define $ H = \left( W,F \right)  $ where $ W:= \left\{ v_0, \dots , v_n \right\}  $ and $ F := \left\{ \left\{ v_i,v_j
\right\}  \mid \left\{ i,j \right\} \in E \right\} \cup \left\{ \left\{ v_0,v_i \right\} \mid i \in \left[ 1:n \right]  \right\} $. 
Let $ f : \mathbb{R} ^{ F  }  \to \mathbb{R} ^{ V \cup E }  $ be the bijective linear map defined by
\begin{align*}
&x_v = z _{ 0v } \text{ for all } v \in V \text{ , and } \\
&y _{ vw } = x_v x_w = \frac{ 1 }{ 2 } \left( z _{ 0v } + z _{ 0w } - z _{ vw }  \right) \text{ for all  } \left\{ v,w \right\} \in E.
\end{align*}  
Then $ P ^{ \text{G}  }  _{ \text{BQP}  } = f \left( P ^{ \text{H}  } _{ \text{CUT}  }  \right)   $.
\end{thm} 
\todo[inline]{This statement is not proven in Mallachs lecture notes, but the paper is:
C. De Simone, “The cut polytope and the boolean quadric polytope”, Discrete Math. 79 (1989), 71–75}
%\textcolor{red}{The Theorem shows that the polytope of \BQP i}

Using this Theorem we can show a fundamental result allowing us to solve \textcolor{red}{BQP} as \textcolor{red}{MC}. 
More specifically the \BQP is of the form:
\begin{cor}
Let $ Q \in \mathbb{R} ^{ n \times n }  $ and $ E = \left\{ \left\{ i,j \right\}  \mid i,j \in \left[ 1:n \right] \text{ such that } q _{ ij } + q _{ ji } \neq 0  \right\}  $.
Then the corresponding \BQP with the objective function $ x ^T Q x $ to be minimised can be solved as a \MCP on $ H = \left( W,F \right)  $ where $ W := \left\{ v_0,
\dots, v_n \right\}  $ and $ F := \left\{ \left\{ v_i,v_j \right\}  \mid \left\{ i,j \right\} \in E \right\} \cup \left\{ \left\{ v_0,v_i \right\}  \mid i \in \left[ 1:n
\right]   \right\}  $, with the objective function:
\begin{align*}
\max \sum_{  \substack{ \left\{ v_i,v_j \right\} \in F \\ v_i \neq v_0 \neq v_j	 }  }^{  } \frac{ 1 }{ 2 } \left( q _{ ij } + q _{ ji }  \right) z _{ ij } - \sum_{ i \in W
\setminus \left\{  v_0  \right\} }^{  } \widetilde{ c } _{ i } z _{ 0i } 
\end{align*} 
where $ \widetilde{ c }_i := q _{ ii } + \sum_{ j=1  }^{ n } \frac{ 1 }{ 2 } \left( q _{ ij } + q _{ ji }  \right) \left[ \left\{ v_i , v_j \right\} \in F, v_i \neq v_0 \neq v_j
\right]  $ for all $ i \in W \setminus \left\{ v_0 \right\}  $ 
\end{cor} 
\begin{proof}
We have the following equality 
\begin{align*}
&x Q x ^T \\
=& \sum_{ i,j = 1 }^{ n } q _{ ij } x_i x_j + \sum_{ i = 1 }^{ n } q _{ ii } x_i x_{ i } \\
\overset{ \overset{ x_i x_i = x_i  }{ \text{def } E} }{ =}&  \sum_{ \left\{ i,j \right\} \in E  }^{  } \left( q _{ ij } + q _{ ji } \right) x_i x_j + \sum_{ i = 1 }^{ n } q
_{ ii } x_i \\
\overset{ \ref{thm:simone}  }{ =}& \sum_{ \left\{ i,j \right\} \in E  }^{  } \left( q _{ ij } + q _{ ji }  \right) \left( \frac{ 1 }{ 2 } \left( z _{ 0i } + z _{ 0j } - z _{
ij }  \right)  \right) + \sum_{ i \in W \setminus \left\{ v_0 \right\}   }^{  } q _{ ii } z _{ 0i } \\
=& -  \sum_{ \left\{ i,j \right\} \in E  }^{  } \frac{ 1 }{ 2 } \left( q _{ ij } + q _{ ji }  \right) z _{ ij } + 
\frac{ 1 }{ 2 } \sum_{  \left\{ i,j \right\}  \in E }^{ } \left( q _{ ij } +
q _{ ji }  \right) \left(  z _{ 0i } + z _{ 0j } \right) + \sum_{ i \in W \setminus \left\{ v_0 \right\}   }^{  } q _{ ii } z _{ 0i } \\
=& - \sum_{ \left\{ i,j \right\} \in E }^{  } \frac{ 1 }{ 2 } \left( q _{ ij } + q _{ ji }  \right) z _{ ij } + 
\sum_{ i \in W \setminus \left\{ v_0 \right\}   }^{  } \underbrace{ \left( q _{ ii } + \frac{ 1 }{ 2 } \sum_{ \overset{ j \neq 0  }{ \left\{ v_i , v_j \right\} \in F} }^{  }  \left( q _{
ij } + q _{ ji }  \right) \right) }_{ =: \tilde{ c } _i }  z _{ 0i } 
\end{align*} 
\todo[inline]{Why does * hold?}
The previous computation together with $  P ^{ \text{G}  }  _{ \text{BQP}  } = f \left( P ^{ \text{H}  } _{ \text{CUT}  }  \right)    $, by Theorem \ref{thm:simone}.
\todo[inline]{Finish this proof: \\
minimising $ xQx^T $ is the same as maximising ...}
\end{proof}

\section{Experimental Setup} 
The goal of our experimental study is to compare the performance of the \BH for a range of different parameters.
We recall that 

\section{Results} 
The experiments gather further data that the \BH is highly effective in returning high quality approximations.
Out of the \textcolor{red}{421} test instances with known optimum value, the instances for which the heuristic \textcolor{red}{returns a cut with ratio less than 0.878 (the
performance guarantee in Goemans Williamson) } are so few we can name them here. They are
\textcolor{red}{name the instances}
In particular the heuristic return high quality cuts for instances with mixed sign weight, these are
\todo[inline]{find out number of instances with mixed sign}
This is very impressive, as the performance guarantee of the Goemans Williamson algorithm only holds for graphs with nonnegative weights. Therefore there is no theoretical
reason for the approximation algorithm to perform well, let alone the \BH.

We point out that the \textcolor{red}{standard deviation} of the returned cut values of the \BH with different parameters is very low in general.

The data indicates, that if a run with \textcolor{red}{big parameters, i.e. expecting longer runtime} finds a very good solution, then a high quality solution can already be
found with small values.
\todo[inline]{What does very good mean?}
\textcolor{red}{
Looking at this from the perspective that the optimal value is not known, the following procedure might be an informed choice: \\
Run the \BH for very small parameters, i.e $ M = 1 $ and $ N = 5 $
}
\bibliography{bibliography}
\bibliographystyle{alpha}
\listoftodos[Notes]
\end{document}
